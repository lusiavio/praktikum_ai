\chapter{NCC}
Please tell more about conclusion and how to the next work of this study.

\section{Lusia Violita Aprilian-1184080}
\subsection{Teori}
\begin{enumerate}
\item Menjelaskan kenapa file teks harus dilakukan tokenizer
	\par Tokenizer adalah untuk membuat vektor dari teks. Dan mengapa harus dilakukan tokenizer? itu karena dengan memfungsikan tokenizer, teks dapat divektorkan. Sehingga teks yang telah telah divektorkan tersebut dapat terbaca pada Machine Learning.
	\par Berikut adalah ilustrasi pemakaian pada tokenizer, perhatikan gambar \ref{7A1}
		\begin{figure}[!hbtp]
		\centering
		\includegraphics[scale=0.4]{figures/v1.jpg}
		\caption{Lusia-Tokenizer}
		\label{7A1}
		\end{figure}

\item Menjelaskan konsep dasar K-Fold Cross Validation
	\lstinputlisting[firstline=8, lastline=20]{src/1164080/7A2.py}
	\par Berdasarkan kode listing tersebut dapat dijelaskan bahwa :
	\begin{enumerate}
	\item Membuat variabel kfold yang memanggil fungsi StratifiedKFold. StratifiedKFold itu sendiri ialah variasi Kfold yang mengembalikan lipatan bertingkat. Yang dimana pada kode program tersebut jumlah lipatannya adalah 5 atau dibagi menjadi 5 bagian.
	\item Membuat variabel split yang mempresentasikan variabel kfold untuk dibagi berdasarkan class.
	\end{enumerate}
	
	\par Berikut adalah gambar ilustrasi dari kosep dasar kfold, perhatikan gambar \ref{7A2}.
		\begin{figure}[!hbtp]
		\centering
		\includegraphics[scale=0.4]{figures/v2.jpg}
		\caption{Lusia-StratifiedKFold}
		\label{7A2}
		\end{figure}

\item Menjelaskan kode program for train, test in splits.
	\lstinputlisting[firstline=8, lastline=20]{src/1164080/7A3.py}
	
	\par Berdasarkan kode program tersebut, maka dapat dijelaskan bahwa kode tersebut digunakan untuk mencetak posisi pengujian pada train dan test yang telah dipisahkan. Dan memastikan bahwa kedua data tersebut tidak terjadi overlaping atau tumpang tindih dalam  setiap pemisahan.
	
	\par Berikut adalah gambar ilustrasi dari for train, test in splits, perhatikan gambar \ref{7A3a}.
	
		\begin{figure}[!hbtp]
		\centering
		\includegraphics[scale=0.4]{figures/v3a.jpg}
		\caption{Lusia-for train, test in splits}
		\label{7A3a}
		\end{figure}
		
	\par Apabila kode program pada gambar \ref{7A3a} dijalankan, maka akan menghasilkan row\_id sebanyak 5 bagian. Dan jika diperthatikan, setiap bagian tidak terjadi pengulangan sehingga bisa bisa dibilang tidak terjadi overlaping pada data. Perhatikan gambar \ref{7A3b}.
		\begin{figure}[!hbtp]
		\centering
		\includegraphics[scale=0.3]{figures/v3b.jpg}
		\caption{Lusia-Hasil Bagian 1}
		\label{7A3b}
		\end{figure}

\item Menjelaskan maksud kode program
	\lstinputlisting[firstline=8, lastline=20]{src/1164080/7A4.py}
	\par Dari kode program tersebut dapat dijelaskan bahwa membuat fungsi train dan test dengan menggunakan dataset yang hanya diambil kolom 'CONTENT' saja. iloc berfingsi sebagai pengindeksan posisi menggunakan integer.
	
	\par Berikut ilustrasi dari kode program tersebut, perhatikan gambar \ref{7A4}.
		\begin{figure}[!hbtp]
		\centering
		\includegraphics[scale=0.4]{figures/v4.jpg}
		\caption{Lusia-Ilustrasi Kode Program No.4}
		\label{7A4}
		\end{figure}

\item Menjelaskan maksud fungsi
	\lstinputlisting[firstline=8, lastline=20]{src/1164080/7A5.py}
		
	\par Dari kode program tersebut dapat dijelaskan bahwa pada baris pertama adalah membuat variabel tokenizer untuk memanggil fungsi tokenizer agar dapat dilakukan vektorisasi dari kata. Dimana pada kode program pada baris pertama menggunakan 2000 kata atau 2000 kolom.
	\par Sedangkan pada baris kedua dari kode program tersebut menjelaskan bahwa tokenizer difungsikan pada data train yang telah di fitting.
	
	\par Berikut adalh gambar ilustrasi dari fungsi pada kode program tersebut, perhatikan gambar \ref{7A5}.
	
		\begin{figure}[!hbtp]
		\centering
		\includegraphics[scale=0.4]{figures/v5.jpg}
		\caption{Lusia-Ilustrasi Maksud fungsi No.5}
		\label{7A5}
		\end{figure}

\item Menjelaskan maksud fungsi
	\lstinputlisting[firstline=8, lastline=20]{src/1164080/7A6.py}
		
	\par Dari fungsi pada kode program tersebut dapat dijelaskan bahwa :
	\begin{enumerate}
	\item Baris pertama membuat variabel d\_train\_inputs untuk memanggil fungsi tokrnizer dan merubah data train yang berupa teks ke dalam bentuk matrix dengan menggunakan model tfidf.
	\item Baris kedua membuat variabel d\_test\_inputs untuk memanggil fungsi tokrnizer dan merubah data test yang berupa teks ke dalam bentuk matrix dengan menggunakan model tfidf.
	\end{enumerate}
	
	\par Berikut adalah gambar ilustrasi dari fungsi pada kode program tersebut, perhatikan gambar \ref{7A6}.
	
		\begin{figure}[!hbtp]
		\centering
		\includegraphics[scale=0.4]{figures/v6.jpg}
		\caption{Lusia-Ilustrasi Maksud fungsi No.6}
		\label{7A6}
		\end{figure}
		
\item Menjelaskan maksud fungsi
	\lstinputlisting[firstline=8, lastline=20]{src/1164080/7A7.py}
		
	\par Dari fungsi pada kode program tersebut dapat dijelaskan bahwa fungsi tersebut akan membagi matrix tfidf tadi dengan amax yaitu mengembalikan maksimum array atau maksimum sepanjang sumbu. Yang hasilnya akan dimasukan kedalam variabel d\_train\_inputs untuk data train dan d\_test\_inputs untuk data test dengan nominal absolut atau tanpa ada bilangan negatif dan koma.
	
	\par Berikut adalah gambar ilustrasi dari fungsi pada kode program tersebut, perhatikan gambar \ref{7A7}.
	
		\begin{figure}[!hbtp]
		\centering
		\includegraphics[scale=0.4]{figures/v7.jpg}
		\caption{Lusia-Ilustrasi Maksud fungsi No.7}
		\label{7A7}
		\end{figure}
	
\item Menjelaskan maksud fungsi
	\lstinputlisting[firstline=8, lastline=20]{src/1164080/7A8.py}
		
	\par Dari fungsi pada kode program tersebut dapat dijelaskan bahwa fungsi tersebut ditujukan untuk melakukan one-hot encoding supaya bisa masuk dan digunakan pada neural network. One-hot encoding diambil dari 'CLASS' yang berarti hanya terdapat 2 neuron, yaitu satu nol(1,0) atau nol satu(0,1) karena pilihannya hanya ada dua (spam atau bukan).
	\par Berikut adalah gambar ilustrasi dari fungsi pada kode program tersebut, perhatikan gambar \ref{7A8}.
		\begin{figure}[!hbtp]
		\centering
		\includegraphics[scale=0.4]{figures/v8.jpg}
		\caption{Lusia-Ilustrasi Maksud fungsi No.8}
		\label{7A8}
		\end{figure}

\item Menjelaskan maksud fungsi
	\lstinputlisting[firstline=8, lastline=20]{src/1164080/7A9.py}
	\par Dari fungsi pada kode program tersebut ditujukan untuk melakukan pemodelan dengan sequential, membandingkan setiap satu larik elemen dengan cara satu persatu secara beruntun. Dimana terdapat 512 neuron inputan dengan input shape 2000 vektor yang sudah dinormalisasi. Lalu model dilakukan aktivasi dengan fungsi 'relu'. Kemudian dilakukan pemotongan bobot supaya tidak overfitting sebesar 50 persen dari neuron inputan 512. Lalu pada layer output terdapat 2 neuron outputan yaitu nol(1,0) atau nol satu(0,1). Kemudian outputan tersebut diaktivasi menggunakan fungsi softmax (mencari nilai maximal).
	\par Berikut adalah gambar ilustrasi dari fungsi pada kode program tersebut, perhatikan gambar \ref{7A9}.
		\begin{figure}[!hbtp]
		\centering
		\includegraphics[scale=0.4]{figures/v9.jpg}
		\caption{Lusia-Ilustrasi Maksud fungsi No.9}
		\label{7A9}
		\end{figure}
		
\item Menjelaskan maksud fungsi
	\lstinputlisting[firstline=8, lastline=20]{src/1164080/7A10.py}
	\par Dari fungsi pada kode program tersebut model yang telah dibuat selanjutnya dicompile dengan menggunakan algoritma optimisasi, fungsi loss, dan fungsi metrik.
	\par Berikut adalah gambar ilustrasi dari fungsi pada kode program tersebut, perhatikan gambar \ref{7A10}.
		\begin{figure}[!hbtp]
		\centering
		\includegraphics[scale=0.4]{figures/v10.jpg}
		\caption{Lusia-Ilustrasi Maksud fungsi No.10}
		\label{7A10}
		\end{figure}

\item Menjelaskan apa itu Deep Learning
	\par Deep Learning merupakan cabang dari Machine Learning atau bagian keluarga yang lebih luas dari method machine learning berdasarkan pada representasi data pembelajaran. Deep Learning menggunakan Deep Neural Network dalam menyelesaikan suatu masalah yang terjadi pada Machine Learning.
	
\item Menjelaskan apa itu Deep Neural dan bedanya dengan Deep Learning
	\par Deep Neural Network atau DNN merupakan algoritma yang berbasis neural network yang digunakan untuk mengambil keputusan.
	\par Yang membedakan Deep Learning dengan  Deep Neural Network (DNN) adalah DNN merupakan algoritma yang digunakan pada Deep Learning, sedangkan Deep Learning merupakan model yang menggunakan algoritma DNN.
	
\item Menjelaskan perhitungan algoritma konvolusi
	\par Konvolusi pada sebuah gambar dilakukan dalam image processing untuk menerapkan operator yang mempunyai nilai output dari piksel gambar yang berasal dari kombinasi linear nilai input piksel tertentu pada gambar. 
	\par Karena NPM saya 1164080 dan hasil dari (NPM mod 3)+1 = 3, maka saya menggunaan matrik kernel berukuran 3x3. Sehingga ilustrasi gambar yang digunakan adalah seperti gambar \ref{7A13}. Misalkan  f(x,y) yang digunakan berukuran 5x5 dan kernel atau mask berukuran 3x3 masing-masing adalah sebagai berikut: 
		\begin{figure}[!hbtp]
		\centering
		\includegraphics[scale=0.4]{figures/v13.jpg}
		\caption{Lusia-Ilustrasi Gambar}
		\label{7A13}
		\end{figure}
	\par Penyelesaian dari operasi konvolusi antara  f(x,y) dengan kernel g(x,y) pada gambar \ref{7A13} adalah  f(x,y) * g(x,y) dengan ilustrasi sebagai berikut :
	\begin{enumerate}
	\item Tempatkan matrik kernel di sebelah kiri atas, lalu hitung nilai piksel pada posisi (0,0) dari kernel tersebut seperti gambar \ref{7A13a}.
		\begin{figure}[!hbtp]
		\centering
		\includegraphics[scale=0.4]{figures/v13a.jpg}
		\caption{Lusia-Ilustrasi konvolusi 1}
		\label{7A13a}
		\end{figure}
		\par Konvolusi dihitung dengan cara berikut :
		\par (0x4) + (-1x4) + (0x3) + (-1x6) + (4x6) + (-1x5) + (0x5) + (-1x6) + (0x6)
		\par Sehingga didapat hasil konvolusi = 3
	\item Lalu geser kernel satu piksel ke kanan kemudian hitung kembali  nilai piksel pada posisi (0,0) dari kernel.
	\par Konvolusi dihitung dengan cara berikut :
	\par (0x4) + (-1x3) + (0x5) + (-1x6) + (4x5) + (-1x5) + (0x6) + (-1x6) + (0x6)  
	\par Sehingga didapat hasil konvolusi = 0 seperti pada gambar \ref{7A13b}.
		\begin{figure}[!hbtp]
		\centering
		\includegraphics[scale=0.4]{figures/v13b.jpg}
		\caption{Lusia-Ilustrasi konvolusi 2}
		\label{7A13b}
		\end{figure}
		
	\item Lalu geser kernel satu piksel ke kanan kemudian hitung kembali  nilai piksel pada posisi (0,0) dari kernel.
	\par Konvolusi dihitung dengan cara berikut :
	\par (0x3) + (-1x5) + (0x4) + (-1x5) + (4x5) + (-1x2) + (0x6) + (-1x6) + (0x2)  
	\par Sehingga didapat hasil konvolusi = 2 seperti pada gambar \ref{7A13c}.
		\begin{figure}[!hbtp]
		\centering
		\includegraphics[scale=0.4]{figures/v13c.jpg}
		\caption{Lusia-Ilustrasi konvolusi 3}
		\label{7A13c}
		\end{figure}
		
	\item Kemudian geser matriks kernel kebawah, lalu mulai hitung kembali dari sisi kiri. Setiap kali perhitungan konvolusi dilakukan, geser matriks kernel atau piksel ke kanan seperti pada gambar \ref{7A13d}, \ref{7A13e}, dan \ref{7A13f}.
	\begin{itemize}
	\item Konvolusi pada gambar \ref{7A13d} dihitung dengan cara berikut :
	\par (0x6) + (-1x6) + (0x5) + (-1x5) + (4x6) + (-1x6) + (0x6) + (-1x7) + (0x5)   
	\par Sehingga didapat hasil konvolusi = 0 seperti pada gambar \ref{7A13d}.
		\begin{figure}[!hbtp]
		\centering
		\includegraphics[scale=0.4]{figures/v13d.jpg}
		\caption{Lusia-Ilustrasi konvolusi 4}
		\label{7A13d}
		\end{figure}
	\item Konvolusi pada gambar \ref{7A13e} dihitung dengan cara berikut :
	\par (0x6) + (-1x5) + (0x5) + (-1x6) + (4x6) + (-1x6) + (0x7) + (-1x5) + (0x5)   
	\par Sehingga didapat hasil konvolusi = 2 seperti pada gambar \ref{7A13e}.
		\begin{figure}[!hbtp]
		\centering
		\includegraphics[scale=0.4]{figures/v13e.jpg}
		\caption{Lusia-Ilustrasi konvolusi 5}
		\label{7A13e}
		\end{figure}
		
	\item Konvolusi pada gambar \ref{7A13f} dihitung dengan cara berikut :
	\par (0x5) + (-1x5) + (0x2) + (-1x6) + (4x6) + (-1x2) + (0x5) + (-1x5) + (0x3)    
	\par Sehingga didapat hasil konvolusi = 6 seperti pada gambar \ref{7A13f}.
		\begin{figure}[!hbtp]
		\centering
		\includegraphics[scale=0.4]{figures/v13f.jpg}
		\caption{Lusia-Ilustrasi konvolusi 6}
		\label{7A13f}
		\end{figure}
	\end{itemize}		
	\item Dengan langkah-langkah yang sama, piksel-piksel pada baris ketiga dilakukan orerasi konvolusi sehingga menghasilkan seperti gambar \ref{7A13g}.
		\begin{figure}[!hbtp]
		\centering
		\includegraphics[scale=0.4]{figures/v13g.jpg}
		\caption{Lusia-Ilustrasi konvolusi baris ketiga}
		\label{7A13g}
		\end{figure}
	\par Apabila nilai piksel hasil konvolusi adalah negatif, maka nilai tersebut dijadikan 0. Sebaliknya, bila nilai piksel hasil konvolusi lebih besar dari nilai kabuan maksimum (255), maka nilai tersebut dijadikan ke nilai keabuan maksimum.
	
	\end{enumerate}

\end{enumerate}

\subsection{Praktek}
\begin{enumerate}
\item Jelaskan kode program pada blok \# In[1]
	\par Berikut adalah kode program yang digunakan :
	\lstinputlisting[firstline=1, lastline=19, caption=Kode Program 1, label={71}]{src/1164080/7B1.py}
	\par Dari kode listing pada kode program 1, dapat dijelaskan seperti berikut :
	\begin{itemize}
	\item Baris 1	: Melakukan pengimportan file csv
	\item Baris 2	: Melakukan pemanggilan atau memasukkan module image sebagai pil\_image dari library PIL
	\item Baris 3	: Melakukan pengimportan fungsi keras.processing.image 
	\end{itemize}
	\par Sehingga dari kode program tersebut bila dijalankan, maka menghasilkan seperti pada gambar \ref{7B1}.
		\begin{figure}[!hbtp]
		\centering
		\includegraphics[scale=0.5]{figures/w1.jpg}
		\caption{Lusia-Hasil Kode Program 1}
		\label{7B1}
		\end{figure}
	
\item Jelaskan kode program pada blok \# In[2]
	\par Berikut adalah kode program yang digunakan :
	\lstinputlisting[firstline=1, lastline=19, caption=Kode Program 2, label={72}]{src/1164080/7B2.py}
	\par Dari kode listing pada kode program 2, dapat dijelaskan seperti berikut :
	\begin{itemize}
	\item Baris 1	: Membuat variabel imgs tanpa ada parameter di dalamnya
	\item Baris 2	: Membuat variabel classes tanpa ada parameter didalamnya
	\item Baris 3	: Membuka file csv dari HASYv2/hasy-data-labels.csv sebagai csvfile
	\item Baris 4	: Membuat variabel csvreader yang difungsikan untuk membaca dari file csv yang dimasukkan
	\item Baris 5	: Membuat variabel i dengan parameter 0 atau nilai 0
	\item Baris 6	: Digunakan untuk melakukan eksekusi baris dari pembacaan csv 
	\item Baris 7	: Mengaplikasikan atau menggunakan perintah "if" dengan variabel i lebih besar dari 0, yang selanjutnya akan dilanjutkan ke perintah berikutnya
	\item Baris 8	: Membuat variabel img yang berfungsi untuk mengubah image atau gambar menjadi bentuk array (bilangan) dari file HASYv2 yang dibuka dengan row berparameter 0.
	\item Baris 9	: Membuat variabel img dengan nilai bukan sama dengan 255.0
	\item Baris 10	: Mendefinisikan fungsi imgs.append yang digunakan untuk melakukan proses penggabungan data dengan file lain atau dataset lain yang telah ditentukan dengan 3 parameter yaitu row[0], row[2] dan variabel img.
	\item Baris 11	: Mendefinisikan fungsi append dari variabel classes dengan menggunakan parameter row[2].
	\item Baris 12	: Mengartikan i=i+1 yang dimana nilai sari variabel i akan ditambah 1 sehingga akan bernilai 1.
	\end{itemize}
	\par Sehingga dari kode program tersebut bila dijalankan, maka menghasilkan seperti pada gambar \ref{7B2}.
		\begin{figure}[!hbtp]
		\centering
		\includegraphics[scale=0.5]{figures/w2.jpg}
		\caption{Lusia-Hasil Kode Program 2}
		\label{7B2}
		\end{figure}
	
\item Jelaskan kode program pada blok \# In[3]
	\par Berikut adalah kode program yang digunakan :
	\lstinputlisting[firstline=1, lastline=19, caption=Kode Program 3, label={73}]{src/1164080/7B3.py}
	\par Dari kode listing pada kode program 3, dapat dijelaskan seperti berikut :
	\begin{itemize}
	\item Baris 1	: Memanggil dan menggunakan module random
	\item Baris 2	: Melakukan pengocokan menggunakan module random pada parameter variabel imgs
	\item Baris 3	: Membagi index data kedalam bentuk integer dengan mengalikan 0,8 dan len yang berfungsi mengembalikan jumlah item dalam datanya dari variabel imgs
	\item Baris 4	: Membuat variabel train yang digunakan untuk mengeksekusi imgs serta pemecahan index awal pada data untuk digunakan sebagai data training
	\item Baris 5	: Membuat variabel test yang digunakan untuk mengeksekusi imgs serta pemecahan index akhir pada data untuk digunakan sebagai data testing
	\end{itemize}
	\par Sehingga dari kode program tersebut bila dijalankan, maka menghasilkan seperti pada gambar \ref{7B3}.
		\begin{figure}[!hbtp]
		\centering
		\includegraphics[scale=0.5]{figures/w3.jpg}
		\caption{Lusia-Hasil Kode Program 3}
		\label{7B3}
		\end{figure}
	
\item Jelaskan kode program pada blok \# In[4]
	\par Berikut adalah kode program yang digunakan :
	\lstinputlisting[firstline=1, lastline=19, caption=Kode Program 4, label={74}]{src/1164080/7B4.py}
	\par Dari kode listing pada kode program 4, dapat dijelaskan seperti berikut :
	\begin{itemize}
	\item Baris 1	: Melakukan import library numpy sebagai np
	\item Baris 2	: Membuat variabel train\_input untuk mengubah inputan menjadi array menggunakan fungsi np.asarray dan  fungsi list untuk mengkoleksi data yang dipilih serta data dapat diubah. Dan didalamnya melakukan penerapan fungsi map yang berfungsi untuk mengembalikan iterator dari data yang digunakan dan fungsi lamda pada row berparameter [2] difungsikan untuk membuat objek menjadi lebih kecil sehingga mudah dieksekusi dari variabel train.
	\item Baris 3	: Membuat variabel test\_input untuk mengubah inputan menjadi array menggunakan fungsi np.asarray dan  fungsi list untuk mengkoleksi data yang dipilih serta data dapat diubah. Dan didalamnya melakukan penerapan fungsi map yang berfungsi untuk mengembalikan iterator dari data yang digunakan dan fungsi lamda pada row berparameter [2] difungsikan untuk membuat objek menjadi lebih kecil sehingga mudah dieksekusi dari variabel test.
	\item Baris 4	: Membuat variabel train\_input untuk mengubah inputan menjadi array menggunakan fungsi np.asarray dan  fungsi list untuk mengkoleksi data yang dipilih serta data dapat diubah. Dan didalamnya melakukan penerapan fungsi map yang berfungsi untuk mengembalikan iterator dari data yang digunakan dan fungsi lamda pada row berparameter [1] difungsikan untuk membuat objek menjadi lebih kecil sehingga mudah dieksekusi dari variabel train.
	\item Baris 5	: Membuat variabel test\_input untuk mengubah inputan menjadi array menggunakan fungsi np.asarray dan  fungsi list untuk mengkoleksi data yang dipilih serta data dapat diubah. Dan didalamnya melakukan penerapan fungsi map yang berfungsi untuk mengembalikan iterator dari data yang digunakan dan fungsi lamda pada row berparameter [1] difungsikan untuk membuat objek menjadi lebih kecil sehingga mudah dieksekusi dari variabel test.
	\end{itemize}
	\par Sehingga dari kode program tersebut bila dijalankan, maka menghasilkan seperti pada gambar \ref{7B4}.
		\begin{figure}[!hbtp]
		\centering
		\includegraphics[scale=0.5]{figures/w4.jpg}
		\caption{Lusia-Hasil Kode Program 4}
		\label{7B4}
		\end{figure}
	
\item Jelaskan kode program pada blok \# In[5]
	\par Berikut adalah kode program yang digunakan :
	\lstinputlisting[firstline=1, lastline=19, caption=Kode Program 5, label={75}]{src/1164080/7B5.py}
	\par Dari kode listing pada kode program 5, dapat dijelaskan seperti berikut :
	\begin{itemize}
	\item Baris 1	: Menggunakan fungsi LabelEncoder dari sklearn.processing yang berfungsi untuk menormalkan label dimana label encoder hanya didefinisikan dengan nilai antara 0 dan -1.
	\item Baris 2	: Menggunakan fungsi OneHotEncoder dari sklearn.processing yang berfungsi untuk mendefinisikan fitur input yang dimana mengambil nilai dalam kisaran [0, nilai maksimal].
	\end{itemize}
	\par Sehingga dari kode program tersebut bila dijalankan, maka menghasilkan seperti pada gambar \ref{7B5}.
		\begin{figure}[!hbtp]
		\centering
		\includegraphics[scale=0.5]{figures/w5.jpg}
		\caption{Lusia-Hasil Kode Program 5}
		\label{7B5}
		\end{figure}
	
\item Jelaskan kode program pada blok \# In[6]
	\par Berikut adalah kode program yang digunakan :
	\lstinputlisting[firstline=1, lastline=19, caption=Kode Program 6, label={75}]{src/1164080/7B6.py}
	\par Dari kode listing pada kode program 6, dapat dijelaskan seperti berikut :
	\begin{itemize}
	\item Baris 1	: Membuat variabel label\_encoder dengan penerapan modul / fungsi LabelEncoder tanpa parameter
	\item Baris 2	: Membuat variabel integer\_encoded dengan penerapan fungsi label\_encoder.fit\_transform yang berfungsi untuk melakukan ekstrasi fitur object dari variabel classes yang akan mengembalikan beberapa data yang diubah kembali.
	\end{itemize}
	\par Sehingga dari kode program tersebut bila dijalankan, maka menghasilkan seperti pada gambar \ref{7B6}.
		\begin{figure}[!hbtp]
		\centering
		\includegraphics[scale=0.5]{figures/w6.jpg}
		\caption{Lusia-Hasil Kode Program 6}
		\label{7B6}
		\end{figure}

\item Jelaskan kode program pada blok \# In[7]
	\par Berikut adalah kode program yang digunakan :
	\lstinputlisting[firstline=1, lastline=19, caption=Kode Program 7, label={77}]{src/1164080/7B7.py}
	\par Dari kode listing pada kode program 7, dapat dijelaskan seperti berikut :
	\begin{itemize}
	\item Baris 1	: Membuat variabel onehot\_encoder yang memanggil fungsi OneHotEncoder tanpa mengembalikan matriks karena sparse=false.
	\item Baris 2	: Membuat variabel integer\_encoded memanggil variabel integer\_encoded pada kode program 6 untuk dieksekusi memberikan bentuk baru ke array tanpa mengubah datanya dari mengembalikan panjang nilai dari integer\_encoded.
	\item Baris 3	: Onehotencoding melakukan fitting pada integer\_encoded.
	\end{itemize}
	\par Sehingga dari kode program tersebut bila dijalankan, maka menghasilkan seperti pada gambar \ref{7B7}.
		\begin{figure}[!hbtp]
		\centering
		\includegraphics[scale=0.5]{figures/w7.jpg}
		\caption{Lusia-Hasil Kode Program 7}
		\label{7B7}
		\end{figure}

\item Jelaskan kode program pada blok \# In[8]
	\par Berikut adalah kode program yang digunakan :
	\lstinputlisting[firstline=1, lastline=19, caption=Kode Program 8, label={78}]{src/1164080/7B8.py}
	\par Dari kode listing pada kode program 8, dapat dijelaskan seperti berikut :
	\begin{itemize}
	\item Baris 1	: Membuat variabel train\_output\_int yang mengeksekusi label\_encoder dengan mengubah nilai dari parameter variabel train\_output.
	\item Baris 2	: Membuat variabel train\_output yang mengeksekusi variabel onehot\_encoder dari kode program 7 dengan mengubah nilai dari variabel parameter train\_output\_int yang datanya sudah diubah kedalam bentuk array dan panjang nilai dari train\_output\_int telah dikembalikan.
	\item Baris 3	: Membuat variabel test\_output\_int yang mengeksekusi label\_encoder dengan mengubah nilai dari parameter variabel test\_output.
	\item Baris 4	: Membuat variabel test\_output yang mengeksekusi variabel onehot\_encoder dari kode program 7 dengan mengubah nilai dari variabel parameter test\_output\_int yang datanya sudah diubah kedalam bentuk array dan panjang nilai dari test\_output\_int telah dikembalikan.
	\item Baris 5	: Membuat variabel num\_classes untuk mengetahui jumlah class dari lebel\_encoder
	\item Baris 6	: Perintah print digunakan untuk memunculkan hasil dari variabel num\_classes
	\end{itemize}
	\par Sehingga dari kode program tersebut bila dijalankan, maka menghasilkan seperti pada gambar \ref{7B8}.
		\begin{figure}[!hbtp]
		\centering
		\includegraphics[scale=0.5]{figures/w8.jpg}
		\caption{Lusia-Hasil Kode Program 8}
		\label{7B8}
		\end{figure}
		
\item Jelaskan kode program pada blok \# In[9]
	\par Berikut adalah kode program yang digunakan :
	\lstinputlisting[firstline=1, lastline=19, caption=Kode Program 9, label={79}]{src/1164080/7B9.py}
	\par Dari kode listing pada kode program 9, dapat dijelaskan seperti berikut :
	\begin{itemize}
	\item Baris 1	: Memanggil atau melakukan importing fungsi model sequential dari library keras.
	\item Baris 2	: Memanggil atau melakukan importing fungsi layer dense, dropout, dan flatten dari library keras.
	\item Baris 3	: Memanggil atau melakukan importing fungsi layer Conv2D dan MaxPooling2D dari library keras.
	\end{itemize}
	\par Sehingga dari kode program tersebut bila dijalankan, maka menghasilkan seperti pada gambar \ref{7B9}.
		\begin{figure}[!hbtp]
		\centering
		\includegraphics[scale=0.5]{figures/w9.jpg}
		\caption{Lusia-Hasil Kode Program 9}
		\label{7B9}
		\end{figure}

\item Jelaskan kode program pada blok \# In[10]
\par Berikut adalah kode program yang digunakan :
	\lstinputlisting[firstline=1, lastline=19, caption=Kode Program 10, label={710}]{src/1164080/7B10.py}
	\par Dari kode listing pada kode program 10, dapat dijelaskan seperti berikut :
	\begin{itemize}
	\item Baris 1	: Melakukan pemodelan Sequential.
	\item Baris 2	: Menambahkan Konvolusi 2D dengan 32 filter konvolusi menggunakan matriks kernel berukuran 3x3 dengan algoritma activation adalah relu, dimana data diambil dari train\_input yang dimulai dari baris ke-nol.
	\item Baris 3	: Menambahkan Max Pooling dengan matriks 2x2.
	\item Baris 4	: Melakukan penambahkan konvolusi 2D lagi dengan 32 filter konvolusi menggunakan matrik kernel masing-masing berukuran 3x3 dengan algoritam activation relu.
	\item Baris 5	: Melakukan penambahan lagi Max Pooling dengan matriks 2x2.
	\item Baris 6	: Melakukan perataan pada inputan model.
	\item Baris 7	: Mendefinisikan inputan dengan 1024 neuron dan melakukan aktivasi menggunakan algoritma tanh.
	\item Baris 8	: Dropout dilakukan untuk melakukan pemotongan pada cabang yang terdiri dari pengaturan secara acak tingkat pecahan unit input ke 0 pada setiap pembaruan selama waktu pelatihan, yang membantu mencegah overfitting sebesar 50\%.
	\item Baris 9	: Pada output layer menggunakan data dari variabel num\_classes dan menggunakan fungsi aktivasi softmax.
	\item Baris 10 	: Melkukan konfigurasi proses pembelajaran, yang dilakukan melalui metode compile, sebelum melatih suatu model.
	\item Baris 11	: Menampilkan hasil representasi ringkasan dari model yang telah dibuat.
	\end{itemize}
	\par Sehingga dari kode program tersebut bila dijalankan, maka menghasilkan seperti pada gambar \ref{7B10}.
		\begin{figure}[!hbtp]
		\centering
		\includegraphics[scale=0.5]{figures/w10.jpg}
		\caption{Lusia-Hasil Kode Program 10}
		\label{7B10}
		\end{figure}

\item Jelaskan kode program pada blok \# In[11]
	\par Berikut adalah kode program yang digunakan :
	\lstinputlisting[firstline=1, lastline=19, caption=Kode Program 11, label={711}]{src/1164080/7B11.py}
	\par Dari kode listing pada kode program 11, dapat dijelaskan seperti berikut :
	\begin{itemize}
	\item Baris 1	: Melakukan importing library keras.callbacks yang memiliki fungsi penulisan log untuk TensorBoard, yang memungkinkan untuk memvisualisasikan grafik dinamis dari training dan metrik pengujian.
	\item Baris 2	: Membuat variabel tenserboard yang menggunakan fungsi TensorBoard dari keras.callbacks yang berfungsi sebagai alat visualisasi yang telah disediakan oleh TensorFlow. Kemudian untuk fungsi log\_dir memanggil data yaitu './logs/mnist-style' dari direktori.
	\end{itemize}
	\par Sehingga dari kode program tersebut bila dijalankan, maka menghasilkan seperti pada gambar \ref{7B11}.
		\begin{figure}[!hbtp]
		\centering
		\includegraphics[scale=0.5]{figures/w11.jpg}
		\caption{Lusia-Hasil Kode Program 11}
		\label{7B11}
		\end{figure}

\item Jelaskan kode program pada blok \# In[12]
	\par Berikut adalah kode program yang digunakan :
	\lstinputlisting[firstline=1, lastline=19, caption=Kode Program 12, label={712}]{src/1164080/7B12.py}
	\par Dari kode listing pada kode program 12, dapat dijelaskan seperti berikut :
	\begin{itemize}
	\item Baris 1	: Menerapkan fungsi model.fit yang didalamnya memproses train\_input, train\_output dengan batch\_size, epochs, verbose, validation\_split, dan callbacks.
		\begin{itemize}
		\item Batch\_size merupakan jumlah sampel per pembaharuan sampel dari data yang diolah, sehingga apabila batch\_sizenya tidak ditemukan maka otomatis akan dijadikan nilai 32.
		\item Epochs berfungsi untuk melakukan perulangan dimana perulangan dari berapa kali nilai yang digunakan untuk data, dan jumlahnya ialah 10.
		\item Verbose digunakan sebagai opsi untuk menghasilkan informasi logging dari data yang ditentukan dengan nilai 2.
		\item Validation\_split berfungsi untuk memecah nilai dari perhitungan dengan validasinya sebesar 0,2. (Fraksi data pelatihan untuk digunakan sebagai data validasi).
		\item Callsbacks mengeksekusi tensorboard yang berfungsi untuk memvisualisasikan parameter training, metrik, hiperparameter pada nilai/data yang diproses.
		\end{itemize}	
	\item Baris 2	: Membuat variabel score dengan menggunakan fungsi evaluate dari model yang ada dengan variabel parameter test\_input, tst\_output dan verbose=2 yang berfungsi memprediksi output untuk input yang diberikan dan kemudian menghitung fungsi metrik yang ditentukan dalam modelnya.
	\item Baris 3	: Mencetak variabel score optimasi dari test dengan ketentuan nilai parameter 0
	\item Baris 4	: Mencetak variabel score akurasi dari test dengan ketentuan nilai parameter 1
	\end{itemize}
	\par Sehingga dari kode program tersebut bila dijalankan, maka menghasilkan seperti pada gambar \ref{7B12}.
		\begin{figure}[!hbtp]
		\centering
		\includegraphics[scale=0.5]{figures/w12.jpg}
		\caption{Lusia-Hasil Kode Program 12}
		\label{7B12}
		\end{figure}

\item Jelaskan kode program pada blok \# In[13]
	\par Berikut adalah kode program yang digunakan :
	\lstinputlisting[firstline=1, lastline=19, caption=Kode Program 13, label={713}]{src/1164080/7B13.py}
	\par Dari kode listing pada kode program 13, dapat dijelaskan seperti berikut :
	\begin{itemize}
	\item Baris 1	: Melakukan importing modul time.
	\item Baris 2	: Membuat variabel result berisikan array kosong.
	\item Baris 3	: Menggunakan convolution 2D yang dimana akan memiliki 1 atau 2 layer
	\item Baris 4	: Mendefinisikan dense\_size dengan ukuran 128, 256, 512, 1024, 2048
	\item Baris 5	: Mendefinsikan drop\_out dengan 0, 25\%, 50\%, dan 75\%
	\item Baris 6	: Melakukan pemodelan Sequential
	\item Baris 7	: Untuk i dalam cakupan conv2d\_count
	\item Baris 8	: Jika ini adalah layer pertama, kita perlu memasukkan bentuk input.
	\item Baris 9	: Kalau tidak kita hanya akan menambahkan layer.
	\item Baris 10	: Kemudian, setelah menambahkan layer konvolusi, akan dilakukan hal yang sama dengan max pooling.
	\item Baris 11	: Lalu, melakukan perataan atau flatten dan menambahkan dense size ukuran apa pun yang berasal dari dense\_size. Dimana akan selalu menggunakan algoritma tanh.
	\item Baris 12	: Jika dropout digunakan, maka akan menambahkan layer dropout. Katakanlah 50\% dilakukan dropout, bahwa setiap kali ia memperbarui bobot setelah setiap batch, ada peluang 50\% untuk setiap bobot yang tidak akan diperbarui dan menempatkannya di antara dua lapisan padat untuk diaktivasi serta melindunginya dari overfitting.
	\item Baris 13	:  Lapisan terakhir akan selalu menjadi jumlah kelas karena itu harus, dan menggunakan softmax. Dan dikompilasi dengan cara yang sama.
	\item Baris 14	:  Mengatur direktori log yang berbeda untuk TensorBoard sehingga dapat membedakan konfigurasi yang berbeda.
	\item Baris 15	: Variabel start akan memanggil modul time atau waktu
	\item Baris 16	: Melakukan fit atau compile 
	\item Baris 17	: Melakukan scoring dengan .evaluate yang berfungsi menampilkan data loss dan accuracy dari model
	\item Baris 18	: end merupakan variabel untuk melihat waktu akhir pada saat pemodelan berhasil dilakukan.
	\item Baris 19	: Menampilkan hasil dari run skrip
	\end{itemize}
	\par Sehingga dari kode program tersebut bila dijalankan, maka menghasilkan seperti pada gambar \ref{7B13}.
		\begin{figure}[!hbtp]
		\centering
		\includegraphics[scale=0.5]{figures/w13.jpg}
		\caption{Lusia-Hasil Kode Program 13}
		\label{7B13}
		\end{figure}

\item Jelaskan kode program pada blok \# In[14]
	\par Berikut adalah kode program yang digunakan :
	\lstinputlisting[firstline=1, lastline=19, caption=Kode Program 14, label={714}]{src/1164080/7B14.py}
	\par Dari kode listing pada kode program 14, dapat dijelaskan seperti berikut :
	\begin{itemize}
	\item Baris 1: Membuat model dengan menggunakan pemodelan Sequential
	\item Baris 2: Pada layer pertama melakukan penambahan Convolutio 2D dengan dimensi 32, dan ukuran matriks kernel 3x3 dengan menggunkan fungsi aktivasi relu dan menampilkan input\_shape
	\item Baris 3: Melakukan Max Pooling 2D dengan ukuran matriks 2x2
	\item Baris 4: Pada layer kedua, dilakukan convolusi lagi dengan kriteria yang sama tanpa ada penambahan input, hal ini dilakukan untuk mendapatkan data terbaik
	\item Baris 5: Melakukan Flatten untuk meratakan inputan
	\item Baris 6: Menambahkan dense input sebanyak 128 neuron dengan menggunakan fungsi aktivasi tanh.
	\item Baris 7: Melakukan Dropout sebanyak 50\% untuk menghindari overfitting
	\item Baris 8: Menambahkan dense pada model untuk output dimana layer ini akan menjadi jumlah dari class yang ada.
	\item Baris 9: Melakukan compiling dari model yang telah didefinisikan 
	\item Baris 10: Menampilkan ringkasan dari pemodelan yang dilakukan 
	\end{itemize}
	\par Sehingga dari kode program tersebut bila dijalankan, maka menghasilkan seperti pada gambar \ref{7B14}.
		\begin{figure}[!hbtp]
		\centering
		\includegraphics[scale=0.5]{figures/w14.jpg}
		\caption{Lusia-Hasil Kode Program 14}
		\label{7B14}
		\end{figure}

\item Jelaskan kode program pada blok \# In[15]
\par Berikut adalah kode program yang digunakan :
	\lstinputlisting[firstline=1, lastline=19, caption=Kode Program 15, label={715}]{src/1164080/7B15.py}
	\par Dari kode listing pada kode program 15, dapat dijelaskan seperti berikut :
	\begin{itemize}
	\item Baris 1: Melakukan fit atau fitting (pencocokan data) dengan penggabungan dari data train dan test agar dapat dilakukan pelatihan untuk jaringan pada smeua data yang dimiliki.
	\end{itemize}
	\par Sehingga dari kode program tersebut bila dijalankan, maka menghasilkan seperti pada gambar \ref{7B15}.
		\begin{figure}[!hbtp]
		\centering
		\includegraphics[scale=0.5]{figures/w15.jpg}
		\caption{Lusia-Hasil Kode Program 15}
		\label{7B15}
		\end{figure}

\item Jelaskan kode program pada blok \# In[16]
\par Berikut adalah kode program yang digunakan :
	\lstinputlisting[firstline=1, lastline=19, caption=Kode Program 16, label={716}]{src/1164080/7B16.py}
	\par Dari kode listing pada kode program 16, dapat dijelaskan seperti berikut :
	\begin{itemize}
	\item Baris 1	: Menyimpan model sebagai mathsymbols.model
	\end{itemize}
	\par Sehingga dari kode program tersebut bila dijalankan, maka menghasilkan seperti pada gambar \ref{7B16}.
		\begin{figure}[!hbtp]
		\centering
		\includegraphics[scale=0.5]{figures/w16.jpg}
		\caption{Lusia-Hasil Kode Program 16}
		\label{7B16}
		\end{figure}

\item Jelaskan kode program pada blok \# In[17]
	\par Berikut adalah kode program yang digunakan :
	\lstinputlisting[firstline=1, lastline=19, caption=Kode Program 17, label={717}]{src/1164080/7B17.py}
	\par Dari kode listing pada kode program 17, dapat dijelaskan seperti berikut :
	\begin{itemize}
	\item Baris 1	: Menyimpan array dari label\_encoder.classes\_ ke file biner dalam format NumPy .npy.
	\end{itemize}
	\par Sehingga dari kode program tersebut bila dijalankan, maka menghasilkan seperti pada gambar \ref{7B17}.
		\begin{figure}[!hbtp]
		\centering
		\includegraphics[scale=0.5]{figures/w17.jpg}
		\caption{Lusia-Hasil Kode Program 17}
		\label{7B17}
		\end{figure}

\item Jelaskan kode program pada blok \# In[18]
	\par Berikut adalah kode program yang digunakan :
	\lstinputlisting[firstline=1, lastline=19, caption=Kode Program 18, label={718}]{src/1164080/7B18.py}
	\par Dari kode listing pada kode program 18, dapat dijelaskan seperti berikut :
	\begin{itemize}
	\item Baris 1	: Melakukan importing modul models dari library keras
	\item Baris 2	: Membuat variabel model2 untuk memanggil dan membaca file mathsymbols.model
	\item Baris 3	: Berfungsi untuk menampilkan dan memeriksa hasil dari variabel parameter model2
	\end{itemize}
	\par Sehingga dari kode program tersebut bila dijalankan, maka menghasilkan seperti pada gambar \ref{7B18}.
		\begin{figure}[!hbtp]
		\centering
		\includegraphics[scale=0.5]{figures/w18.jpg}
		\caption{Lusia-Hasil Kode Program 18}
		\label{7B18}
		\end{figure}

\item Jelaskan kode program pada blok \# In[19]
	\par Berikut adalah kode program yang digunakan :
	\lstinputlisting[firstline=1, lastline=19, caption=Kode Program 19, label={719}]{src/1164080/7B19.py}
	\par Dari kode listing pada kode program 19, dapat dijelaskan seperti berikut :
	\begin{itemize}
	\item Baris 1	: Membuat variabel label\_encoder2 yang melakukan pemanggilan fungsi LabelEncoder untuk melakukan penyandian pada label dengan nilai antara 1 dan 0
	\item Baris 2	: Variabel label\_encoder akan memanggil class yang telah disimpan sebelumnya.
	\item Baris 3	: Function Predict akan mengubah gambar kedalam bentuk array
	\item Baris 4	: Variabel prediction akan melakukan prediksi untuk model2 dengan Memberikan bentuk baru tanpa mengubah data dari variabel newimg dengan bentuk array 4D.
	\item Baris 5	: Variabel inverted akan mencari nilai tertinggi output dari hasil prediksi.
	\end{itemize}
	\par Sehingga dari kode program tersebut bila dijalankan, maka menghasilkan seperti pada gambar \ref{7B19}.
		\begin{figure}[!hbtp]
		\centering
		\includegraphics[scale=0.5]{figures/w19.jpg}
		\caption{Lusia-Hasil Kode Program 19}
		\label{7B19}
		\end{figure}
		
\item Jelaskan kode program pada blok \# In[20]
	\par Berikut adalah kode program yang digunakan :
	\lstinputlisting[firstline=1, lastline=19, caption=Kode Program 20, label={720}]{src/1164080/7B20.py}
	\par Dari kode listing pada kode program 20, dapat dijelaskan seperti berikut :
	\begin{itemize}
	\item Baris 1	: Melakukan prediksi terhadap file v2-00010.png yang diambil dari direktori
	\item Baris 2	: Melakukan prediksi terhadap file v2-00500.png yang diambil dari direktori
	\item Baris 3	: Melakukan prediksi terhadap file v2-00700.png yang diambil dari direktori
	\end{itemize}
	\par Sehingga dari kode program tersebut bila dijalankan, maka menghasilkan seperti pada gambar \ref{7B20}.
		\begin{figure}[!hbtp]
		\centering
		\includegraphics[scale=0.5]{figures/w20.jpg}
		\caption{Lusia-Hasil Kode Program 20}
		\label{7B20}
		\end{figure}

\end{enumerate}

\subsection{Penanganan Error}
\begin{enumerate}
	\item skrinsut error 
		\begin{enumerate}
		\item Error 1 : gambar \ref{7C1}
		\begin{figure}[!hbtp]
		\centering
		\includegraphics[scale=0.5]{figures/x1.jpg}
		\caption{Lusia-skrinsut error 1}
		\label{7C1}
		\end{figure}
		\item Error 2 : gambar \ref{7C2}
		\begin{figure}[!hbtp]
		\centering
		\includegraphics[scale=0.5]{figures/x2.jpg}
		\caption{Lusia-skrinsut error 2}
		\label{7C2}
		\end{figure}
		\end{enumerate}
	\item Tuliskan kode eror dan jenis errornya
		\begin{enumerate}
		\item Error 1
			\par Berdasarkan gambar \ref{7C1} berikut adalah kode eror dan jenis errornya :
			\begin{itemize}
			\item Kode error : ModuleNotFoundError: No module named 'tensorflow'
			\item Jenis error : Module Not Found Error
			\end{itemize}
		\item Error 2
			\par Berdasarkan gambar \ref{7C2} berikut adalah kode eror dan jenis errornya :
			\begin{itemize}
			\item Kode error : FileNotFoundError: [Errno 2] No such file or directory: 'HASYv2/hasy-data-labels.csv'
			\item Jenis error : File Not Found Error
			\end{itemize}
		\end{enumerate}
	\item Solusi pemecahan masalah error tersebut
		\begin{enumerate}
		\item Error 1
			\par Berdasarkan gambar \ref{7C1} error yang terjadi akibat modul tensorflow belum diinstal, maka penyelesaiannya adalah dengan menginstal terlebih dahulu modul yang akan digunakan.
		\item Error 2
			\par Berdasarkan gambar \ref{7C2} error yang terjadi akibat direktori file yang akan digunakan tidak sesuai, maka solusinya dengan menyesuaikan direktori dari file yang akan digunakan.
		\end{enumerate}
\end{enumerate}

\subsection{Cek Plagiarisme}
\par Dari hasil kerja pada chapter 7, jika dicek plagiarisme menghasilkan seperti gambar \ref{7D1}.
		\begin{figure}[!hbtp]
		\centering
		\includegraphics[scale=0.4]{figures/pc7.jpg}
		\caption{Lusia-Plagiarisme}
		\label{7D1}
		\end{figure}




\section{Rahmi Roza-1164085}
\subsection{Teori}
\begin{enumerate}
\item Kenapa File Suara Harus Dilakukan Tokenizer
\begin{itemize}
\item Penjelasan: Untuk membedakan karakter-karakter tertentu dalam suatu teks dan juga sebagai pemisah kata atau bukan.Tokenizer dilakukan dengan cara melakukan pemotongan string input berdasarkan tiap kata yang menyusunnya.
\par 
\par
\item Ilustrasi Gambar
\item Tokenizer \ref{teori1}
\begin{figure}[!hbtp]
\centering
\includegraphics[scale=0.7]{figures/teori1.jpg}
\caption{Tokenizer Roza}
\label{teori1}
\end{figure}
\par
\end{itemize}
\par
\par

\item Jelaskan konsep dasar K Fold Cross Validation pada dataset komentar Youtube pada kode listing 
\lstinputlisting[firstline=8, lastline=20]{src/1164085/chapter7-2.py}
\begin{itemize}
\item Penjelasan: Startified KFold berisikan presentasi sampel untuk setiap kelas. Dimana dalam ilustrasi ini sampel dibagi menjadi 5 dalam setiap class nya. Kemudian sampel tadi akan dimasukan kedalam class dari dataset youtube tadi.
\par 
\par
\item Ilustrasi Gambar
\item K-Fold Cross Validation \ref{teori2}
\begin{figure}[!hbtp]
\centering
\includegraphics[scale=0.7]{figures/teori2.jpg}
\caption{K-Fold Cross Validation Roza}
\label{teori2}
\end{figure}
\par
\end{itemize}
\par
\par

\item Jelaskan apa maksudnya kode program for train, test in splits.dilengkapi dengan ilustrasi atau gambar.
\begin{itemize}
\item Penjelasan: Maksudnya yaitu untuk menguji apakah setiap data pada dataset sudah di split dan tidak terjadi penumpukan. Yang dimana maksudnya di setiap class tidak akan muncul id yang sama. Ilustrasinya misalkan kita memiliki 4 botol minuman dengan model yang berbeda. Kemudian kita bagikan kedua anak, tentunya setiap anak yang menerima botol tidak memiliki botol minuman  yang sama modelnya.
\par 
\par
\item Ilustrasi Gambar
\item No 3  \ref{teori3}
\begin{figure}[!hbtp]
\centering
\includegraphics[scale=0.2]{figures/teori3.png}
\caption{No 3 Roza}
\label{teori3}
\end{figure}
\par
\end{itemize}
\par
\par

\item Jelaskan apa maksudnya kode program {train\_content = d['CONTENT'].iloc[train\_idx]} dan {test\_content = d['CONTENT'].iloc[test\_idx]}.
\begin{itemize}
\item Penjelasan: Maksudnya yaitu mengambil data pada kolom atau index CONTENT yang merupakan bagian dari train\_idx dan test\_idx. Ilustrasinya, ketika data telah diubah menjadi train dan test maka kita dapat memilihnya untuk ditampilkan pada kolom yang diinginkan.
\par 
\par
\end{itemize}
\par
\par

\item Jelaskan apa maksud dari fungsi tokenizer {tokenizer = Tokenizer(num\_words=2000)} dan tokenizer.{tokenizer.fit\_on\_texts(train\_content)}, dilengkapi dengan ilustrasi atau gambar.
\begin{itemize}
\item Penjelasan: Dimana variabel tokenizer akan melakukan vektorisasi kata menggunakan fungsi Tokenizer yang dimana jumlah kata yang ingin diubah kedalam bentuk token adalah 2000 kata. Dan untuk \emph{tokenizer.fit\_on\_texts(train\_content)} maksudnya kita akan melakukan fit tokenizer hanya untuk data trainnya saja tidak dengan data test nya untuk kolom CONTENT. 
\par 
\par
\item Ilustrasi Gambar
\item No 5 \ref{teori5}
\begin{figure}[!hbtp]
\centering
\includegraphics[scale=0.4]{figures/teori5.png}
\caption{No 5 Roza}
\label{teori5}
\end{figure}
\par
\end{itemize}
\par
\par


\item Jelaskan apa maksud dari fungsi \emph{d\_train\_inputs = tokenizer.texts\_to\_matrix(train\_content, mode='tfidf')} dan \emph{d\_test\_inputs = tokenizer.texts\_to\_matrix(test\_content, mode='tfidf')}, dilengkapi dengan ilustrasi kode dan atau gambar.
\begin{itemize}
\item Penjelasan: Maksudnya yaitu untuk variabel d\_train\_inputs akan melakukan tokenizer dari bentuk teks ke matrix dari data train\_content dengan mode matriksnya yaitu tfidf begitu juga dengan variabel d\_test\_inputs untuk data test. Berikut gambar ilustrasinya
\par 
\par
\item Ilustrasi Gambar
\item No 6 \ref{teori6}
\begin{figure}[!hbtp]
\centering
\includegraphics[scale=0.3]{figures/teori6.jpg}
\caption{No 6 Roza}
\label{teori6}
\end{figure}
\par
\end{itemize}
\par
\par

\item Jelaskan apa maksud dari fungsi \emph{d\_train\_inputs = d\_train\_inputs/np.amax(np.absolute(d\_train\_inputs))} dan \emph{d\_test\_inputs = d\_test\_inputs/np.amax(np.absolute(d\_test\_inputs))}, dilengkapi dengan ilustrasi atau gambar.
\begin{itemize}
\item Penjelasan: Fungsi tersebut akan membagi matrix tfidf tadi dengan amax yaitu mengembalikan maksimum array atau maksimum sepanjang sumbu. Yang hasilnya akan dimasukan kedalam variabel d\_train\_inputs untuk data train dan d\_test\_inputs untuk data test dengan nominal absolut atau tanpa ada bilangan negatif dan koma.
\par 
\par
\item Ilustrasi Gambar
\item No 7 \ref{teori7}
\begin{figure}[!hbtp]
\centering
\includegraphics[scale=0.4]{figures/teori7.png}
\caption{No 7 Roza}
\label{teori7}
\end{figure}
\par
\end{itemize}
\par
\par

\item Jelaskan apa maksud fungsi dari d\_train \_outputs = np\_utils.to\_categorical(d['CLASS'].iloc[train dan d\_test \_outputs = np\_ utils.to\_categorical(d['CLASS'].iloc[test\_idx]) dalam kode program
\begin{itemize}
\item Penjelasan: Dari fungsi pada kode program tersebut dijelaskan fungsi tersebut ditujukan untuk melakukan one-hot encoding agar dapat masuk dan digunakan di neural network. One-hot encoding diambil dari 'CLASS' yang berarti hanya terdapat 2 neuron, yaitu satu nol(1,0) atau nol satu(0,1) karena pilihannya hanya ada dua (spam atau bukan).
\par 
\par
\item Ilustrasi Gambar
\item No 8 \ref{teori8}
\begin{figure}[!hbtp]
\centering
\includegraphics[scale=0.4]{figures/teori8.png}
\caption{No 8 Roza}
\label{teori8}
\end{figure}
\par
\end{itemize}
\par
\par

\item Jelaskan apa maksud dari fungsi di listing!
\lstinputlisting[firstline=8, lastline=20]{src/1164085/chapter7-9.py}
\begin{itemize}
\item Penjelasan: Dari fungsi pada kode program tersebut ditujukan untuk melakukan pemodelan dengan sequential, membandingkan setiap satu larik elemen dengan cara satu persatu secara beruntun. Dimana terdapat 512 neuron inputan dengan input shape 2000 vektor. Lalu model dilakukan aktivasi dengan fungsi 'relu'. Kemudian dilakukan pemotongan bobot supaya tidak overfitting sebesar 50 persen dari neuron inputan 512. Lalu pada layer output terdapat 2 neuron outputan yaitu nol(1,0) atau nol satu(0,1). Kemudian outputan tersebut diaktivasi menggunakan fungsi softmax.
\par 
\end{itemize}
\par
\par

\item Jelaskan apa maksud dari fungsi di listing!
\lstinputlisting[firstline=8, lastline=20]{src/1164085/chapter7-10.py}
\begin{itemize}
\item Penjelasan: Dari fungsi pada kode program tersebut model yang telah dibuat selanjutnya dicompile dengan menggunakan algoritma optimisasi denganfungsi fungsi loss, fungsi opttimaizer dan fungsi metrik. Dimana nama masing-masing fungsi tersebut adalah categorical\_crossentropy, adamax dan accuracy.
\par 
\end{itemize}
\par
\par

\item 	Apa Itu Deep Learning
\begin{itemize}
\item Penjelasan: 
\par  Deep learning merupakan sub bidang pembelajaran mesin yang berkaitan dengan algoritma.
\end{itemize}
\par
\par

\item Apa itu Deep Neural Network Dan Apa Bedanya Dengan Deep Learning :
\begin{itemize}
\item Penjelasan Deep Neural Network : 
\par  Deep neural network adalah jaringan syaraf dengan tingkat kompleksitas tertentu, jaringan syaraf dengan lebih dari dua lapisan.
\par
\item Perbedaan Deep Neural Network Dan Deep Learning :
\par Perbedaan antara deep neural network dan deep learning terletak pada kedalaman model. deep learning adalah frasa yang digunakan untuk jaringan saraf yang kompleks. Kompleksitas ini disebabkan oleh pola yang rumit tentang bagaimana informasi dapat mengalir di seluruh model.
\end{itemize}
\par
\par

\item Perhitungan Algoritma Konvolusi Dengan NPM 1164085. Seperti gambar di bawah penyelesaiannya:
\begin{itemize}
\item Ilustrasi Gambar
\item No 13 \ref{teori13}
\begin{figure}[!hbtp]
\centering
\includegraphics[scale=0.6]{figures/teori13.jpg}
\caption{No 13 Roza}
\label{teori13}
\end{figure}
\par
\end{itemize}
\par
\par
\item Plagiarisme Roza
\begin{figure}[!hbtp]
\centering
\includegraphics[scale=0.6]{figures/plagairismerozac7.jpg}
\caption{Plagiarisme Roza}
\label{teori13}
\end{figure}
\par
\end{enumerate}
\par
\par

\subsection{Praktek}
\begin{enumerate}
\item Jelaskan kode program pada blok \# In[1].
\begin{itemize}
\item Kode Program:
\lstinputlisting[firstline=1, lastline=19, caption=Praktek1.py, label={lst:import}]{src/1164085/prak1-roza.py}
\par Hasil \ref{in1roza} :
\begin{figure}[!hbtp]
\centering
\includegraphics[scale=0.7]{figures/prak1roza.jpg}
\caption{In 1 Roza}
\label{in1roza}
\end{figure}
\par Baris 1: Mengimpoer file csv.
\par Baris 2: Dari library PIL impor gambar sebagai pil\_image.
\par Baris 3: Impor fungsi fungsi keras.processing.image
\end{itemize}
\par

\item Jelaskan kode program pada blok \# In[2].
\begin{itemize}
\item Kode Program:
\lstinputlisting[firstline=1, lastline=19, caption=Praktek2.py, label={lst:import}]{src/1164085/prak2-roza.py}
\par Hasil \ref{in2roza} :
\begin{figure}[!hbtp]
\centering
\includegraphics[scale=0.7]{figures/prak2roza.jpg}
\caption{In 2 Roza}
\label{in2roza}
\end{figure}
\par Baris 1: Membuat variabel images tanpa parameter, karena di dalam kurung kosong tanpa ada parameter yang diisi.
\par Baris 2: Membuat variabel class tanpa parameter juga karna tiada ada parameter di dalam kurung.
\par Baris 3: Membuka file HASYv2/hasy-data-labels.csv sebagai csvfile.
\par Baris 4: Membuat variabel csvreader yang memfungsikan pembacaan dari file csv yang dimasukkan
\par Baris 5: Membuat variabel i dengan parameter 0
\par Baris 6: Mengeksekusi data dari baris di pembacaan file csv.
\par Baris 7: Menggunakan perintah "if" dengan ketentuan variabel i lebih besar dari  0.
\par Baris 8: Membuat variabel img dengan nama keras.processing yang mengubah image menjadi bentuk array (bilangan) dari file HASYv2 yang dibuka dengan row berparameter 0.
\par Baris 9: Membuat variabel img tidak sama dengan 255.0
\par Baris 10: Mendefinisikan fungsi imgs.append dimana merupakan proses menggabungkan data dengan file lain  yang ditentukan dengan 3 parameter yaitu row[0], row[2] dan variabel img.
\par Baris 11: Mendefinisikan fungsi append kembali dari variabel classes dengan parameternya row[2].
\par Baris 12: Mendefinisikan fungsi dimana i variabel i akan ditambah nilainya sehingga akan bernilai 1 ( Contoh nilai i=0 dengan adanya penambahan maka hasilnya akan menjadi 1 )
\end{itemize}
\par


\item Jelaskan kode program pada blok \# In[3].
\begin{itemize}
\item Kode Program:
\lstinputlisting[firstline=1, lastline=19, caption=Praktek3.py, label={lst:import}]{src/1164085/prak3-roza.py}
\par Hasil \ref{in3roza} :
\begin{figure}[!hbtp]
\centering
\includegraphics[scale=0.7]{figures/prak3roza.jpg}
\caption{In 3 Roza}
\label{in3roza}
\end{figure}
\par Baris 1: Import modul random
\par Baris 2: Melakukan pengocokan terhadap modul dengan parameter dimana variabelnya imgs
\par Baris 3: Membagi dan memecah index dalam bentuk integer dengan mengkalikan nilai 0.8 dengan fungsi len yang mengembalikan jumlah item dari variabel imgs.
\par Baris 4: Membuat variabel train yang mengeksekusi imgs dengan pemecahan index pada data awa.l
\par Baris 5: Membuat variabel test yang mengeksekusi imgs dengan pemecahan index pada dataakhir.

\end{itemize}
\par

\item Jelaskan kode program pada blok \# In[4].
\begin{itemize}
\item Kode Program:
\lstinputlisting[firstline=1, lastline=19, caption=Praktek4.py, label={lst:import}]{src/1164085/prak4-roza.py}
\par Hasil \ref{in4roza} :
\begin{figure}[!hbtp]
\centering
\includegraphics[scale=0.7]{figures/prak4roza.jpg}
\caption{In 4 Roza}
\label{in4roza}
\end{figure}
\par Baris 1: Import library numpy sebagai np.
\par Baris 2: Membuat variabel train\_input dimana mengubah input menjadi sebuah array dari np dengan menggunakan fungsi list untuk mengkoleksikan data yang dipilih dan dapat diubah. Didalamnya diterapkan fungsi map untuk mengembalikan iterator dari datanya dengan memfungsikan lamda pada row dengan parameter [2] untuk membuat objek fungsi menjadi lebih kecil dan mudah dieksekusi dari variabel train.
\par Baris 3:Membuat variabel test\_input dengan fungsi yang sama seperti train\_input yang membedakan hanya datanya / inputan yang diproses berasal dari variabel test
\par Baris 4: Membuat variabel train\_output dimana mengubah keluaran menjadi sebuah array dari np dengan menggunakan fungsi list untuk mengkoleksikan data yang dipilih dan dapat diubah. Didalamnya diterapkan fungsi map untuk mengembalikan iterator dari datanya dengan memfungsikan lamda pada row dengan parameter[1] untuk membuat objek fungsi menjadi lebih kecil dan mudah dieksekusi dari variabel train.
\par Baris 5: Membuat variabel test\_output dengan fungsi yang sama seperti train\_output yang membedakan hanya datanya / inputan yang diproses berasal dari variabel test
\end{itemize}
\par

\item Jelaskan kode program pada blok \# In[5].
\begin{itemize}
\item Kode Program:
\lstinputlisting[firstline=1, lastline=19, caption=Praktek5.py, label={lst:import}]{src/1164085/prak5-roza.py}
\par Hasil \ref{in5roza} :
\begin{figure}[!hbtp]
\centering
\includegraphics[scale=0.7]{figures/prak5roza.jpg}
\caption{In 5 Roza}
\label{in5roza}
\end{figure}
\par Baris 1: Import labelEncoder dari sklearn.processing digunakan untuk menormalkan label dimana label encoder hanya didefinisikan dengan nilai antara 0 dan n\_classes-1.
\par Baris 2: Memasukkan modul / fungsi OneHotEncoder dari sklearn.processing yang digunakan untuk mendefinisikan fitur input dimana mengambil nilai dalam kisaran [0, maks (nilai)).
\end{itemize}
\par

\item Jelaskan kode program pada blok \# In[6].
\begin{itemize}
\item Kode Program:
\lstinputlisting[firstline=1, lastline=19, caption=Praktek5.py, label={lst:import}]{src/1164085/prak6-roza.py}
\par Hasil \ref{in6roza} :
\begin{figure}[!hbtp]
\centering
\includegraphics[scale=0.7]{figures/prak6roza.jpg}
\caption{In 6 Roza}
\label{in6roza}
\end{figure}
\par Baris 1: Membuat variabel label\_encoder dengan penerapan modul / fungsi dari LabelEncoder tanpa parameter
\par Baris 2: Membuat variabel integer\_encoded dengan penerapan fungsi label\_encoder.fit\_transform (ekstrasi fitur object ) dari variabel classes dimana akan mengembalikan beberapa data yang diubah kembali dari variabel label\_encoder.
\end{itemize}
\par

\item Jelaskan kode program pada blok \# In[7].
\begin{itemize}
\item Kode Program:
\lstinputlisting[firstline=1, lastline=19, caption=Praktek7.py, label={lst:import}]{src/1164085/prak7-roza.py}
\par Hasil \ref{in7roza} :
\begin{figure}[!hbtp]
\centering
\includegraphics[scale=0.7]{figures/prak7roza.png}
\caption{In 7 Roza}
\label{in6roza}
\end{figure}
\par Baris 1: Membuat variabel onehot\_encoder yang memanggil fungsi OneHotEncoder tanpa mengembalikan matriks karena sparse=false.
\par Baris 2: Membuat variabel integer\_encoded memanggil variabel integer\_encoded pada kode program 6 untuk dieksekusi memberikan bentuk baru ke array tanpa mengubah datanya dari mengembalikan panjang nilai dari integer\_encoded.
\par Baris 3: Onehotencoding melakukan fitting pada integer\_encoded.
\end{itemize}
\par

\item Jelaskan kode program pada blok \# In[8].
\begin{itemize}
\item Kode Program:
\lstinputlisting[firstline=1, lastline=19, caption=Praktek8.py, label={lst:import}]{src/1164085/prak8-roza.py}
\par Hasil \ref{in8roza} :
\begin{figure}[!hbtp]
\centering
\includegraphics[scale=0.7]{figures/prak8roza.png}
\caption{In 8 Roza}
\label{in8roza}
\end{figure}
\par Baris 1: Membuat variabel train\_output\_int yang mengeksekusi label\_encoder dengan mengubah nilai dari parameter variabel train\_output.
\par Baris 2: Membuat variabel train\_output yang mengeksekusi variabel onehot\_encoder dari kode program 7 dengan mengubah nilai dari variabel parameter train\_output\_int yang datanya sudah diubah kedalam bentuk array dan panjang nilai dari train\_output\_int telah dikembalikan.
\par Baris 3: Membuat variabel test\_output\_int yang mengeksekusi label\_encoder dengan mengubah nilai dari parameter variabel test\_output.
\par Baris 4: Membuat variabel test\_output yang mengeksekusi variabel onehot\_encoder dari kode program 7 dengan mengubah nilai dari variabel parameter test\_output\_int yang datanya sudah diubah kedalam bentuk array dan panjang nilai dari test\_output\_int telah dikembalikan.
\par Baris 5: Membuat variabel num\_classes untuk mengetahui jumlah class dari lebel\_encoder
\par Baris 6: Perintah print digunakan untuk memunculkan hasil dari variabel num\_classes
\end{itemize}
\par

\item Jelaskan kode program pada blok \# In[9].
\begin{itemize}
\item Kode Program:
\lstinputlisting[firstline=1, lastline=19, caption=Praktek9.py, label={lst:import}]{src/1164085/prak9-roza.py}
\par Hasil \ref{in9roza} :
\begin{figure}[!hbtp]
\centering
\includegraphics[scale=0.7]{figures/prak9roza.png}
\caption{In 9 Roza}
\label{in9roza}
\end{figure}
\par Baris 1: Memanggil atau melakukan importing fungsi model sequential dari library keras.
\par Baris 2: Memanggil atau melakukan importing fungsi layer dense, dropout, dan flatten dari library keras.
\par Baris 3: Memanggil atau melakukan importing fungsi layer Conv2D dan MaxPooling2D dari library keras.
\end{itemize}
\par

\item Jelaskan kode program pada blok \# In[10].
\begin{itemize}
\item Kode Program:
\lstinputlisting[firstline=1, lastline=19, caption=Praktek10.py, label={lst:import}]{src/1164085/prak10-roza.py}
\par Hasil \ref{in10roza} :
\begin{figure}[!hbtp]
\centering
\includegraphics[scale=0.7]{figures/prak10roza.png}
\caption{In 10 Roza}
\label{in10roza}
\end{figure}
\par Baris 1: 
\par Baris 2:
\par Baris 3:
\end{itemize}
\par

\item Jelaskan kode program pada blok \# In[11].
\begin{itemize}
\item Kode Program:
\lstinputlisting[firstline=1, lastline=19, caption=Praktek11.py, label={lst:import}]{src/1164085/prak11-roza.py}
\par Hasil \ref{in11roza} :
\begin{figure}[!hbtp]
\centering
\includegraphics[scale=0.7]{figures/prak11roza.png}
\caption{In 11 Roza}
\label{in11roza}
\end{figure}
\par Baris 1: Memasukkan / Mengimport library keras.callbacks dimana digunakan dalam penulisan log untuk TensorBoard, yang memungkinkan untuk memvisualisasikan grafik dinamis dari pelatihan dan metrik pengujian.
\par Baris 2: Membuat variabel tenserboard yang mendefinisikan fungsi TensorBoard pada keras.callbacks yang digunakan sebagai alat visualisasi yang disediakan dengan TensorFlow. Kemudian untuk fungsi log\_dir (jalur direktori tempat menyimpan file log yang akan diuraikan oleh TensorBoard) memanggil data yaitu './logs/mnist-style'
\end{itemize}
\par

\item Jelaskan kode program pada blok \# In[12].
\begin{itemize}
\item Kode Program:
\lstinputlisting[firstline=1, lastline=19, caption=Praktek12.py, label={lst:import}]{src/1164085/prak12-roza.py}
\par Hasil \ref{in12roza} :
\begin{figure}[!hbtp]
\centering
\includegraphics[scale=0.7]{figures/prak12roza.png}
\caption{In 12 Roza}
\label{in12roza}
\end{figure}
\par Baris 1: Menerapkan fungsi model.fit yang didalamnya memproses train\_input, train\_output
\par Baris 2: Selanjutnya pada penerapan fungsi yang sama difungsikan batch\_size ( jumlah sampel per pembaharuan sampel dari data yang diolah) apabila batch\_sizenya tidak ditemukan maka otomatis akan dijadikan nilai 32	
\par BAris 3: Pada penerapan fungsi yang sama, difungsikan epochs dimana perulangan dari berapa kali nilai yang digunakan untuk data, dan jumlahnya ialah 10
\par Baris 4: Mendefinisikan fungsi verbose dimana digunakan sebagai opsi untuk menghasilkan informasi logging dari data yang ditentukan dengan nilai 2
\par Baris 5: Mendefinisikan fungsi validation\_split untuk memecah nilai dari perhitungan validasinya sebesar 0,2. (Fraksi data pelatihan untuk digunakan sebagai data validasi)
\par Baris 6: Mendefinisikan fungsi callsbacks dengan parameternya yang mengeksekusi tensorboard dimana digunakan untuk visualisasikan parameter training, metrik, hiperparameter pada nilai/data yang diproses
\par Baris 7: Mendefinisikan variabel score dengan fungsi evaluate dari model yang ada dengan parameter test\_input, tst\_output dan verbose=2 dimana memprediksi output untuk input yang diberikan dan kemudian menghitung fungsi metrik yang ditentukan dalam modelnya
\item Baris 8: Mencetak score optimasi dari test dengan ketentuan nilai parameter 0
\item Baris 9: Mencetak score akurasi dari test dengan ketentuan nilai parameter 1
\end{itemize}
\par

\item Jelaskan kode program pada blok \# In[13].
\begin{itemize}
\item Kode Program:
\lstinputlisting[firstline=1, lastline=19, caption=Praktek13.py, label={lst:import}]{src/1164085/prak13-roza.py}
\par Hasil \ref{in13roza} :
\begin{figure}[!hbtp]
\centering
\includegraphics[scale=0.7]{figures/prak13roza.png}
\caption{In 13 Roza}
\label{in13roza}
\end{figure}
\par Baris 1: impor modul time dari python anaconda
\par Baris 2: Variabel result berisikan array kosong.
\par Baris 3: Menggunakan convolution 2D yang dimana akan memiliki 1 atau 2 layer.
\par Baris 4: Mendefinisikan dense\_size dengan ukuran 128, 256, 512, 1024, 2048
\par Baris 5: Mendefinsikan drop\_out dengan 0, 25\%, 50\%, dan 75\%
\par Baris 6:  Melakukan pemodelan Sequential
\par Baris 7: Jika ini adalah layer pertama, kita perlu memasukkan bentuk input.
\par Baris 8: Kalau tidak kita hanya akan menambahkan layer.
\par Baris 9: Kemudian, setelah menambahkan layer konvolusi, kita akan melakukan hal yang sama dengan max pooling.
\par Baris 10: Lalu, kita akan meratakan atau flatten dan menambahkandense size ukuran apa pun yang berasal dari dense\_size. Dimana akan selalu menggunakan algoritma tanh
\par Baris 11: Jika dropout digunakan, kita akan menambahkan layer dropout. Menyebut dropout ini berarti, katakanlah 50\%, bahwa setiap kali ia memperbarui bobot setelah setiap batch, ada peluang 50\% untuk setiap bobot yang tidak akan diperbarui
\par Baris 12: menempatkan ini di antara dua lapisan padat untuk dihidupkan dari melindunginya dari overfitting.
\par Lapisan terakhir akan selalu menjadi jumlah kelas karena itu harus, dan menggunakan softmax. Itu dikompilasi dengan cara yang sama.
\par Baris 13: Atur direktori log yang berbeda untuk TensorBoard sehingga dapat membedakan konfigurasi yang berbeda.
\par Baris 14: Variabel start akan memanggil modul time atau waktu
\par Baris 15: Melakukan fit atau compile 
\par Baris 16: Melakukan scoring dengan .evaluate yang akan menampilkan data loss dan accuracy dari model
\par Baris 17:  end merupakan variabel untuk melihat waktu akhir pada saat pemodelan berhasil dilakukan.
\par Baris 18:  Menampilkan hasil dari run skrip diatas 
\end{itemize}
\par

\item Jelaskan kode program pada blok \# In[14].
\begin{itemize}
\item Kode Program:
\lstinputlisting[firstline=1, lastline=19, caption=Praktek14.py, label={lst:import}]{src/1164085/prak14-roza.py}
\par Hasil \ref{in14roza} :
\begin{figure}[!hbtp]
\centering
\includegraphics[scale=0.7]{figures/prak14roza.png}
\caption{In 14 Roza}
\label{in14roza}
\end{figure}
\par Baris 1: Melakukan pemodelan Sequential
\par Baris 2: Untuk layer pertama, Menambahkan Convolutio 2D dengan dmensi 32, dan ukuran matriks 3x3 dengan function aktivasi yang digunakan yaitu relu dan menampilkan input\_shape
\par Baris 3: Dilakukan Max Pooling 2D dengan ukuran matriks 2x2
\par Baris 4:Untuk layer kedua, melakukan Convolusi lagi dengan kriteria yang sama tanpa menambahkan input, ini dilakukan untuk mendapatkan data yang terbaik
\par Baris 5:  Flatten digubakan ntuk meratakan inputan
\par Baris 6: Menambahkan dense input sebanyak 128 neuron dengan menggunakan function aktivasi tanh.
\par Baris 7: Dropout sebanyak 50\% untuk menghindari overfitting
\par Baris 8: Menambahkan dense pada model untuk output dimana layer ini akan menjadi jumlah dari class yang ada.
\par Baris 9: Mengcompile model yang didefinisikan diatas
\par Baris 10: Menampilkan ringkasan dari pemodelan yang dilakukan
\end{itemize}
\par

\item Jelaskan kode program pada blok \# In[15].
\begin{itemize}
\item Kode Program:
\lstinputlisting[firstline=1, lastline=19, caption=Praktek15.py, label={lst:import}]{src/1164085/prak15-roza.py}
\par Hasil \ref{in15roza} :
\begin{figure}[!hbtp]
\centering
\includegraphics[scale=0.7]{figures/prak15roza.png}
\caption{In 15 Roza}
\label{in15roza}
\end{figure}
\par Baris 1: Melakukan fit dengan join data train dan test agar dapat dilakukan pelatihan untuk jaringan pada smeua data yang dimiliki.
\end{itemize}
\par

\item Jelaskan kode program pada blok \# In[16].
\begin{itemize}
\item Kode Program:
\lstinputlisting[firstline=1, lastline=19, caption=Praktek1.py, label={lst:import}]{src/1164085/prak16-roza.py}
\par Hasil \ref{in16roza} :
\begin{figure}[!hbtp]
\centering
\includegraphics[scale=0.7]{figures/prak16roza.png}
\caption{In 16 Roza}
\label{in16roza}
\end{figure}
\par Baris 1: Menyimpan atau save model yang telah di latih dengan nama mathsymbols.model 
\end{itemize}
\par

\item Jelaskan kode program pada blok \# In[17].
\begin{itemize}
\item Kode Program:
\lstinputlisting[firstline=1, lastline=19, caption=Praktek1.py, label={lst:import}]{src/1164085/prak17-roza.py}
\par Hasil \ref{in17roza} :
\begin{figure}[!hbtp]
\centering
\includegraphics[scale=0.7]{figures/prak17roza.png}
\caption{In 17 Roza}
\label{in17roza}
\end{figure}
\par Baris 1: Simpan label enkoder (untuk membalikkan one-hot encoder) dengan nama classes.npy
\end{itemize}
\par

\item Jelaskan kode program pada blok \# In[18].
\begin{itemize}
\item Kode Program:
\lstinputlisting[firstline=1, lastline=19, caption=Praktek18.py, label={lst:import}]{src/1164085/prak18-roza.py}
\par Hasil \ref{in18roza} :
\begin{figure}[!hbtp]
\centering
\includegraphics[scale=0.7]{figures/prak18roza.png}
\caption{In 18 Roza}
\label{in18roza}
\end{figure}
\par Baris 1: Impor models dari librari Keras
\par Baris 2: Variabel model2 akan memanggil model yang telah disave tadi 
\par Baris 3:  Menampilkan ringkasan dari hasil pemodelan
\end{itemize}
\par


\item Jelaskan kode program pada blok \# In[19].
\begin{itemize}
\item Kode Program:
\lstinputlisting[firstline=1, lastline=19, caption=Praktek19.py, label={lst:import}]{src/1164085/prak19-roza.py}
\par Hasil \ref{in19roza} :
\begin{figure}[!hbtp]
\centering
\includegraphics[scale=0.7]{figures/prak19roza.png}
\caption{In 19 Roza}
\label{in19roza}
\end{figure}
\item Menampilkan hasil dari variabel prediction dan inverted
\par Baris 1: Memanggil fungsi LabelEncoder
\par Baris 2: Variabel label\_encoder akan memanggil class yang disave sebelumnya.
\par Baris 3: Function Predict akan mengubah gambar kedalam bentuk array
\par Baris 4: Variabel prediction akan melakukan prediksi untuk model2 dengan reshape variabel newimg dengan bentukarray 4D.
\par Baris 5: Variabel inverted akan mencari nilai tertinggi output dari hasil prediksi tadi
\end{itemize}
\par

\item Jelaskan kode program pada blok \# In[20].
\begin{itemize}
\item Kode Program:
\lstinputlisting[firstline=1, lastline=19, caption=Praktek20.py, label={lst:import}]{src/1164085/prak20-roza.py}
\par Hasil \ref{in20roza} :
\begin{figure}[!hbtp]
\centering
\includegraphics[scale=0.7]{figures/prak20roza.png}
\caption{In 20 Roza}
\label{in20roza}
\end{figure}
\par Baris 1: Melakukan prediksi dari pelatihan dari gambar v2-00010.png
\par Baris 2: Melakukan prediksi dari pelatihan dari gambar v2-00500.png
\par Baris 3: Melakukan prediksi dari pelatihan dari gambar v2-00700.png 
\end{itemize}
\par

\subsection{Penanganan Error}
\begin{enumerate}
\lstinputlisting[firstline=1, lastline=19, caption=Eror Capter7 Roza.py, label={lst:import}]{src/1164085/erorchap7roza.py}
\item Skrinsut Error
\begin{figure}[ht]
\centering
\includegraphics[scale=0.7]{figures/erorchap7roza.jpg}
\caption{ Error Roza}
\label{6}
\end{figure}
\item Kode Error dan Jenis Errornya
\par Kode Error: "Name Error" name 'imgs'  is not defined
\par Jenis Error: Images tidak terdefinisi
\item Penanganan
\par Menentukan atau memebuka file explorer dari file yang ditempatkan atau disesuaikan dengan letak filenya.

\end{enumerate}
\end{enumerate}








\section{Fadila-1164072}
\subsection{Teori}
Penjelasan Tugas Harian 12 ( No 1-11 ).
\begin{enumerate}
\item Mengapa File Teks Harus Dilakukan Tokenizer Besera Ilustrasi Gambar :
\begin{itemize}
\item Tokenizer :
\par Difungsikan untuk membuat vektor dari text. Lebih detailnya, tokenizer merupakan langkah pertama yang diperlukan dalam banyak tugas pemrosesan bahasa alami, seperti penghitungan kata, penguraian, pemeriksaan ejaan, pembuatan corpus, dan analisis statistik teks.
\par
\par
\item Mengapa Text Harus Dilakukan Tokenizer ? :
\par Text harus dilakukan tokenizer agar dapat dirubah menjadi vektor. Dari perubahan ke vektor tersebut maka data/textnya dapat dibaca oleh komputer (terkomputerisasi).
\par
\par
\item Ilustrasi Gambar ( Contoh ): \ref{chapter-7-tokenizer-fadila}
\par
\begin{figure}[!hbtp]
\centering
\includegraphics[scale=0.2]{figures/chapter-7-tokenizer-fadila.png}
\caption{Tokenizer - fadila}
\label{chapter-7-tokenizer-fadila}
\end{figure}
\par
\end{itemize}
\par
\par
\item Konsep Dasar K Fold Cross Validation Pada Dataset Komentar Youtube Pada Kode Listing 1 Beserta Dengan Ilustrasi Gambar :
\begin{itemize}
\item Code		:
\lstinputlisting[firstline=8, lastline=20,caption=K Fold Cross Validation,label={lst:7.0}]{src/1164072/teori/chapter-7-2-fadila.py}
\item Penjelasan	: 
\par Untuk kejelasan dari StartifiedKFold yang dicontohkan ialah digunakan dan berisikan presentasi sampel untuk setiap kelas yang ada ( youtube). Pada ilusrasinya sampel dibagi menjadi 5 dalam setiap kelas yang diproses. Kemudian sampelnya sendiri akan dimasukan kedalam class dari dataset youtube yang digunakan.
\par
\item Ilustrasi Gambar ( Contoh ): \ref{chapter-7-starfied-k-fold-cross-fadila}
\par
\par
\begin{figure}[!hbtp]
\centering
\includegraphics[scale=0.2]{figures/chapter-7-starfied-k-fold-cross-fadila.jpg}
\caption{Startified K-Fold Cross - fadila}
\label{chapter-7-starfied-k-fold-cross-fadila}
\end{figure}
\par
\par
\end{itemize}
\par
\par
\par
\item Jelaskan Apa Maksud Kode Program For Train Dan Test In Splits Dilengkapi Dengan Ilustrasi Gambar :
\begin{itemize}
\item Penjelasan	:
\par Kode Program For Train dan Test In Splits sendiri digunakan ataupun difungsikan untuk pengujian. Pegujiannya yaitu menguji apakah setiap data pada dataset yang dieksekusi sudah di split dan tidak terjadi penumpukan pada data tersebut. Dengan tidak terjadinya penumpukan seperti hal tersebut maka setiap class nya tidak akan memunculkan id yang sama.
\par 
\par
\par
\item Ilustrasi Kalimat ( Contoh ):
\par Dimisalkan dengan contoh dimana terdapat 6 meja dengan model / desain yang berbeda.Langkah selanjutnya dilakukan pendistribusian terhadap meja-meja tersebut kepada 3 investor ( penjual ) maka setiap penjual tersebut tidak akan menjual model meja yang sama dipasaran.
\par
\item Ilustrasi Gambar ( Contoh Lainnya ): \ref{chapter-7-test-in-splits-fadila}
\par
\begin{figure}[!hbtp]
\centering
\includegraphics[scale=0.2]{figures/chapter-7-test-in-splits-fadila.jpg}
\caption{chapter-7-test-in-splits-fadila}
\label{chapter-7-test-in-splits-fadila}
\end{figure}
\par
\par
\end{itemize}
\par
\par
\par
\item Apa Maksud Kode Program\emph{train\_content = d['CONTENT'].iloc[train\_idx]} dan \emph{test\_content = d['CONTENT'].iloc[test\_idx]}. Dilengkapi Dengan Ilustrasi Gambar :
\begin{itemize}
\item Penjelasan	:
\par Maksud dari code program tersebut ialah difungsikan dalam pengambilan data pada kolom atau index CONTENT. index Content tersebut merupakan bagian dari train\_idx dan test\_idx. Dengan pendefinisian dan pengaplikasian index Content tersebut pada bagian training data dan test data akan dilakukan pengambilan data yang sesuai dan sama.
\par
\par
\item Ilustrasi Kalimat ( Contoh ):
\par Jika dicontohkan dengan sebuah penjelasan maka bisa dikatakan apabila index CONTENT tersebut direalisasikan maka untuk data yang telah diubah menjadi training data maupun test data bisa dipilih kolom ataupun nilai apa yang akan ditampilkan dan dieksekusi sesuai dengan kebetuhan ataupun keinginan.
\par
\par 
\item Ilustrasi Gambar ( Contoh Lainnya ): \ref{chapter-7-iloc-fadila}
\par Apabila ingin dicontohkan dengan contoh yang lain, dapat dilihat pada gambar berikut dimana contoh ini lebih sederhana hanya menjelaskan tentang penerapan d.iloc yang lebih simple sehingga apabila diterapkan dengan parameter / fungsi lain seperti content dapat lebih dimudah untuk dikerjakan ( berdasarkan basicnya )
\par
\begin{figure}[!hbtp]
\centering
\includegraphics[scale=0.45]{figures/chapter-7-iloc-fadila.jpg}
\caption{Contoh Penerapan .Iloc- fadila}
\label{chapter-7-iloc-fadila}
\end{figure}
\par
\par
\end{itemize}
\par
\par
\par
\item Apa Maksud Dari Fungsi \emph{Tokenizer = Tokenizer(num words=2000) Dan Tokenizer.fit on texts(train content)}, Dilengkapi Dengan Ilustrasi Gambar :
\begin{itemize}
\item Penjelasan	:
\par  Fungsi dari Tokenizer diatas ialah untuk melakukan vektorisasi kata tentunya. Fungsi tokenizer ini mengeksekusi jumlah data yang akan diubah sebesar 2000 kata. 
\par Kemudian untuk  \emph{tokenizer.fit\_on\_texts(train\_content)} digunakan untuk melakukan fit tokenizer. Fungsi tersebut hanya direalisasikan pada data train dan tidak untuk data test kemudian hanya dalam / untuk kolom content seperti yang diperlihatkan pada codingannya.
\item Ilustrasi Kalimat ( Contoh ):
\par Jika dicontohkan dengan sebuah penjelasan maka bisa dikatakan apabila ada kata seperti " My Name Is Far " , " My Name Is ", " Your Name Is ", jika dibuatkan atau dijalankan dengan perintah yang dikatan sama maka akan nampak hasilnya seperti { 'far': 4, 'is' : 1, 'my' : 3, 'name' : 2, 'your':5 } .
\par
\par
\par
\item Ilustrasi Gambar ( Contoh Lainnya ): \ref{chapter-7-tokenizer-sum-words-fadila}
\par
\par
\begin{figure}[!hbtp]
\centering
\includegraphics[scale=0.2]{figures/chapter-7-tokenizer-sum-words-fadila.jpg}
\caption{Tokenizer Fit On Text- fadila}
\label{chapter-7-tokenizer-sum-words-fadila}
\end{figure}
\par
\par
\end{itemize}
\par
\par
\par
\item Apa Maksud Dari Fungsi code berikut ( \emph{d\_train\_inputs = tokenizer.texts\_to\_matrix(train\_content, mode='tfidf')} dan \emph{d\_test\_inputs = tokenizer.texts\_to\_matrix(test\_content, mode='tfidf')} ), Dilengkapi Dengan Ilustrasi Kode Dan Atau Gambar :
\begin{itemize}
\item Penjelasan	:
\par Dapat dikatakan bahwa maksud dari codingan diatas ialah untuk variabel d\_train\_inputs dimana akan melakukan tokenizer dari bentuk teks ke / menjadi matrix dari data train\_content menggunakan mode matriks. Mode matriksnya sendiri yaitu tfidf, dan pengaplikasian mode ini juga digunakan untuk dengan variabel d\_test\_inputs untuk data test.
\par
\par
\item Ilustrasi Code Dan Gambar	( Contoh ): \ref{chapter-7-tokenizer-text-matrix-fadila}
\lstinputlisting[firstline=8, lastline=20]{src/1164072/teori/chapter-7-6-fadila.py}
\par
\par
\begin{figure}[!hbtp]
\centering
\includegraphics[scale=0.7]{figures/chapter-7-tokenizer-text-matrix-fadila.jpg}
\caption{Tokenizer Text To Matrix- fadila}
\label{chapter-7-tokenizer-text-matrix-fadila}
\end{figure}
\par
\par
\end{itemize}
\par
\par
\par
\item Jelaskan Apa Maksud Dari Fungsi Berikut ( \emph{d\_train\_inputs = d\_train\_inputs/np.amax(np.absolute(d\_train\_inputs))} dan \emph{d\_test\_inputs = d\_test\_inputs/np.amax(np.absolute(d\_test\_inputs))} ) Kemudian Dilengkapi Dengan Ilustrasi Gambar :
\begin{itemize}
\item Penjelasan : 
\par Berdasarkan code diatas, menjelaskan bahwa fungsi tersebut akan membagi matrix tfidf yang sudah dieksekusi sebelumnya dengan amax. Amaxnya sendiri berfungsi dalam pengembalian maksimum array atau maksimum sepanjang sumbu. Untuk hasilnya akan dimasukan kedalam variabel d\_train\_inputs untuk data train dan d\_test\_inputs untuk data test berupa nilai absolute atautanpa ada bilangan negatif.
\par
\par
\item Ilustrasi Code Dan Gambar	( Contoh ): \ref{chapter-7-np-amax-fadila} dan \ref{chapter-7-np-absolute-fadila}
\begin{itemize}
\item Contoh 1 :
\lstinputlisting[firstline=8, lastline=20]{src/1164072/teori/chapter-7-7-amax-fadila.py}
\par
\par
\par
\begin{figure}[!hbtp]
\centering
\includegraphics[scale=0.7]{figures/chapter-7-np-amax-fadila.jpg}
\caption{Contoh Penerapan Np Amax- fadila}
\label{chapter-7-np-amax-fadila}
\end{figure}
\par
\item Contoh 2 :
\lstinputlisting[firstline=8, lastline=20]{src/1164072/teori/chapter-7-7-absolute-fadila.py}
\par
\par
\begin{figure}[!hbtp]
\centering
\includegraphics[scale=0.7]{figures/chapter-7-np-absolute-fadila.jpg}
\caption{Contoh Penerapan Np Absolute- fadila}
\label{chapter-7-np-absolute-fadila}
\end{figure}
\par
\end{itemize}
\end{itemize}
\par
\par
\item Jelaskan Apa Maksud Dari \emph{d\_train\_outputs = np.utils.to\_categorical(d['CLASS'].iloc[train]} Dan \emph{d\_test\_outputs = np\_utils.to\_categorical(d['CLASS'].iloc[test\_idx]} Dalam Kode Program Dilengkapi Dengan Ilustrasi Gambar :
\begin{itemize}
\item Penjelasan : 
\par Yang dimaksudkan dari kode program tersebut dapat dijelaskan bahwa fungsnya ditujukan untuk melakukan one-hot encoding. One-hot encoding itu direalisasikan supaya bisa masuk dan digunakan pada neural network. 
\par One-hot encoding diambil dari 'CLASS' yang berarti hanya terdapat 2 neuron, yaitu dengan nilai satu nol(1,0) atau nol satu(0,1). Mengapa demikian? dikarenakan pilihannya hanya ada dua yaitu spam atau bukan spam.
\par
\par
\item Ilustrasi Code Dan Gambar	( Contoh ): \ref{chapter-7-np-utils-to-categorical-fadila}
\lstinputlisting[firstline=8, lastline=20]{src/1164072/teori/chapter-7-8-fadila.py}
\par
\begin{figure}[!hbtp]
\centering
\includegraphics[scale=0.6]{figures/chapter-7-np-utils-to-categorical-fadila.jpg}
\caption{Contoh Penerapan Np.utils.to categorical - fadila}
\label{chapter-7-np-utils-to-categorical-fadila}
\end{figure}
\par
\par
\end{itemize}
\par
\par
\par
\item Jelaskan Maksud Dari Fungsi Di Listing 7.2. Gambarkan Ilustrasi Neural Networknya Dari Model Kode Tersebut.
\begin{itemize}
\item Code :
\lstinputlisting[firstline=8, lastline=20]{src/1164072/teori/chapter-7-9-fadila.py}
\item Penjelasan : 
\par Berdasarkan code tersebut, dimaksudkan atau ditujukan untuk melakukan pemodelan dengan sequential. Sequential tersebut untuk membandingkan setiap satu larik elemen dengan cara satu persatu secara beruntun. Terdapat 512 neuron inputan dengan input shape 2000 vektor yang sudah dinormalisasi. 
\par Selanjutnya model dilakukan aktivasi dengan fungsi 'relu', yang setelahnya terjadi pemotongan bobot  sebesar 50 persen dari neuron inputan 512 ( agar tidak overfitting ). Pada layer output terdapat 2 neuron outputan yaitu nol(1,0) atau nol satu(0,1). Setelah semua ketentuan tersebut, dimunculkan outputan yang diaktivasi menggunakan fungsi softmax.
\par
\par
\end{itemize}
\par
\par
\par
\par
\item Jelaskan Maksud Dari Fungsi Di Listing 7.3. Dengan Parameter Berikut :
\begin{itemize}
\item Code :
\lstinputlisting[firstline=8, lastline=20]{src/1164072/teori/chapter-7-10-fadila.py}
\item Penjelasan : 
\par Berdasarkan code tersebut , dimaksudkan bahwa model yang telah dibuat akan dicompile dengan menggunakan algoritma optimisasi (adamax), fungsi loss(categorical\_crossentropy), dan fungsi metrik untuk perhitungan akurasinya.
\par
\par
\end{itemize}
\par
\par
\par
\item Apa itu Deep Learning :
\begin{itemize}
\item Penjelasan :
\par Deep learning merupakan sub bidang pembelajaran mesin yang berkaitan dengan algoritma yang terinspirasi oleh struktur dan fungsi otak yang disebut jaringan saraf tiruan.
\par
\par
\par
\end{itemize}
\item Apa itu Deep Neural Network Dan Apa Bedanya Dengan Deep Learning :
\begin{itemize}
\item Penjelasan Deep Neural Network : 
\par Deep neural network adalah jaringan syaraf dengan tingkat kompleksitas tertentu, jaringan syaraf dengan lebih dari dua lapisan. Deep neural netwok menggunakan pemodelan matematika yang canggih untuk memproses data dengan cara yang kompleks.
\par
\item Perbedaan Deep Neural Network Dan Deep Learning :
\par Perbedaan antara deep neural network dan deep learning terletak pada kedalaman model. deep learning adalah frasa yang digunakan untuk jaringan saraf yang kompleks. Kompleksitas ini disebabkan oleh pola yang rumit tentang bagaimana informasi dapat mengalir di seluruh model. Arsitekturnya menjadi lebih kompleks tetapi konsep deep learning masih sama. Meskipun sekarang ada peningkatan jumlah layer dan node tersembunyi yang terintegrasi untuk memperkirakan output.
\par Untuk pemahaman yang lebih baik, diberikan sebuah contoh terkait dengan penjelasan perbedaan antara deep neural network dan deep learning.
\par
\par
\item Ilustrasi Gambar ( Contoh ): \ref{chapter-7-beda-deep-neu-dan-deep-learn-fadila}
\par
\par
\begin{figure}[!hbtp]
\centering
\includegraphics[scale=0.2]{figures/chapter-7-beda-deep-neu-dan-deep-learn-fadila.jpg}
\caption{Perbedaan Deep NW Dan Deep Learn- fadila}
\label{chapter-7-beda-deep-neu-dan-deep-learn-fadila}
\end{figure}
\par
\par
\end{itemize}
\par
\par
\item Bagaimana Perhitungan Algoritma Dengan Ukuran Stride (NPM mod3+1)x(NPM mod3+1) Yang  Terdapat Pada Max Pooling :
\begin{itemize}
\item Penjelasan :
\par Konvolusi pada sebuah gambar dilakukan dalam image processing untuk menerapkan operator yang mempunyai nilai output dari piksel gambar yang berasal dari kombinasi linear nilai input piksel tertentu pada gambar.
\par
\item Langkah-langkah Algoritma Konvulasi Sesuai NPM : \ref{chapter-7-algoritma-konvolusi-fadila}
\par
\par
\begin{figure}[!hbtp]
\centering
\includegraphics[scale=0.52]{figures/chapter-7-algoritma-konvolusi-fadila.jpg}
\caption{Langkah Algoritma Konvolusi Berdasarkan NPM- fadila}
\label{chapter-7-algoritma-konvolusi-fadila}
\end{figure}
\par
\par
\end{itemize}
\end{enumerate}

\begin{itemize}
\item  Plagiarisme Fadila: \ref{chapter-7-plagiarisme-fadila}
\par
\begin{figure}[!hbtp]
\centering
\includegraphics[scale=0.3]{figures/chapter-7-plagiarisme-fadila.jpg}
\caption{Plagiarisme- fadila}
\label{chapter-7-plagiarisme-fadila}
\end{figure}
\par
\par
\end{itemize}


\par
\par
\par
\par
\subsection{Praktek}
Penjelasan Tugas Harian 12 ( No 1-20 ).
\begin{enumerate}
\item Jelaskan Kode Program Pada Blok \# In[1]. Jelaskan Arti Dari Setiap Baris Kode Yang Dibuat Dan Hasil Keluarannya Dari Komputer Sendiri.
\begin{itemize}
\item Code Yang Digunakan : \ref{lst:chapter-7-1-fadila}.
\lstinputlisting[firstline=1, lastline=19,caption=File chapter-7-1-fadila.py, label={lst:chapter-7-1-fadila}]{src/1164072/praktek/chapter-7-1-fadila.py}
\par
\par
\item Penjelasan Code Perbaris	: 
\begin{enumerate}
\item Baris Code 1	: Memasukkan / Mengimport file csv
\item Baris Code 2	: Memasukkan module image sebagai pil\_image dari library PIL dimana PIL dapat mendukung pembukaan, pemanipulasian, dan menyimpan banyak format file gambar yang berbeda
\item Baris Code 3	: Memasukkan / mengimport fungsi keras.processing.image 
\end{enumerate}
\par
\item Hasil : \ref{chapter-7-in-1-fadila}
\par
\par
\begin{figure}[!hbtp]
\centering
\includegraphics[scale=0.4]{figures/chapter-7-in-1-fadila.jpg}
\caption{Code Program Pada In [1] - fadila}
\label{chapter-7-in-1-fadila}
\end{figure}
\par
\par
\end{itemize}
\par
\par
\par
\item Jelaskan Kode Program Pada Blok \# In[2]. Jelaskan Arti Dari Setiap Baris Kode Yang Dibuat Dan Hasil Keluarannya Dari Komputer Sendiri.
\begin{itemize}
\item Code Yang Digunakan : \ref{lst:chapter-7-2-fadila}.
\lstinputlisting[firstline=1, lastline=19,caption=File chapter-7-2-fadila.py, label={lst:chapter-7-2-fadila}]{src/1164072/praktek/chapter-7-2-fadila.py}
\par
\par
\item Penjelasan Code Perbaris	: 
\begin{enumerate}
\item Baris Code 1	: Membuat variabel imgs tanpa parameter
\item Baris Code 2	: Membuat variabel classes tanpa parameter
\item Baris Code 3	: Membuka file HASYv2/hasy-data-labels.csv sebagai csvfile
\item Baris Code 4	: Membuat variabel csvreader yang memfungsikan pembacaan dari file csv yang dimasukkan
\item Baris Code 5	: Membuat variabel i dengan parameter 0
\item Baris Code 6	: Mengeksekusi baris dari pembacaan csv 
\item Baris Code 7	: Mengaplikasikan perintah "if" dengan ketentuan variabel i lebih besar dari angka 0, maka akan dilanjutkan ke perintah berikutnya
\item Baris Code 8	: Membuat variabel img yang mengubah image menjadi bentuk array (bilangan) dari file HASYv2 yang dibuka dengan row berparameter 0.
\item Baris Code 9	: Membuat variabel img tidak sama dengan nilai 255.0
\item Baris Code 10	: Mendefinisikan fungsi imgs.append dimana merupakan proses melampirkan atau menggabungkan data dengan file lain atau set data yang ditentukan dengan 3 parameter yaitu row[0], row[2] dan variabel img.
\item Baris Code 11	: Mendefinisikan fungsi append kembali dari variabel classes dengan parameternya row[2].
\item Baris Code 12	: Mendefinisikan fungsi dimana i variabel i akan ditambah nilainya sehingga akan bernilai 1 ( 0 dengan penambahan maka akan jadi 1 )
\end{enumerate}
\par
\item Hasil : \ref{chapter-7-in-2-fadila}
\par
\par
\begin{figure}[!hbtp]
\centering
\includegraphics[scale=0.4]{figures/chapter-7-in-2-fadila.jpg}
\caption{Code Program Pada In [2] - fadila}
\label{chapter-7-in-2-fadila}
\end{figure}
\par
\par
\end{itemize}
\par
\par
\par
\item Jelaskan Kode Program Pada Blok \# In[3]. Jelaskan Arti Dari Setiap Baris Kode Yang Dibuat Dan Hasil Keluarannya Dari Komputer Sendiri.
\begin{itemize}
\item Code Yang Digunakan : \ref{lst:chapter-7-3-fadila}.
\lstinputlisting[firstline=1, lastline=19,caption=File chapter-7-3-fadila.py, label={lst:chapter-7-3-fadila}]{src/1164072/praktek/chapter-7-3-fadila.py}
\par
\par
\item Penjelasan Code Perbaris	: 
\begin{enumerate}
\item Baris Code 1	: Memasukkan dan memfungsikan module random
\item Baris Code 2	: Melakukan pengocokan pada module random dengan parameter variabelnya imgs
\item Baris Code 3	: Membagi dan memecah index dalam bentuk integer dengan mengkalikannilai 0,8 dan fungsi len yang akan mengembalikan jumlah item dalam datanya dari variabel imgs
\item Baris Code 4	: Membuat variabel train yang mengeksekusi imgs dengan pemecahan index awal pada data
\item Baris Code 5	: Membuat variabel test yang mengeksekusi imgs dengan pemecahan index akhir pada data
\end{enumerate}
\par
\item Hasil : \ref{chapter-7-in-3-fadila}
\par
\par
\begin{figure}[!hbtp]
\centering
\includegraphics[scale=0.4]{figures/chapter-7-in-3-fadila.jpg}
\caption{Code Program Pada In [3] - fadila}
\label{chapter-7-in-3-fadila}
\end{figure}
\par
\par
\end{itemize}
\par
\par
\par
\item Jelaskan Kode Program Pada Blok \# In[4]. Jelaskan Arti Dari Setiap Baris Kode Yang Dibuat Dan Hasil Keluarannya Dari Komputer Sendiri.
\begin{itemize}
\item Code Yang Digunakan : \ref{lst:chapter-7-4-fadila}.
\lstinputlisting[firstline=1, lastline=19,caption=File chapter-7-4-fadila.py, label={lst:chapter-7-4-fadila}]{src/1164072/praktek/chapter-7-4-fadila.py}
\par
\par
\item Penjelasan Code Perbaris	: 
\begin{enumerate}
\item Baris Code 1	: Memasukkan / Mengimport library numpy sebagai np
\item Baris Code 2	: Membuat variabel train\_input dimana mengubah input menjadi sebuah array dari np dengan menggunakan fungsi list untuk mengkoleksikan data yang dipilih dan dapat diubah. Didalamnya diterapkan fungsi map untuk mengembalikan iterator dari datanya dengan memfungsikan lamda pada row dengan parameter [2] untuk membuat objek fungsi menjadi lebih kecil dan mudah dieksekusi dari variabel train.
\item Baris Code 3	: Membuat variabel test\_input dengan fungsi yang sama seperti train\_input yang membedakan hanya datanya / inputan yang diproses berasal dari variabel test
\item Baris Code 4	: Membuat variabel train\_output dimana mengubah keluaran menjadi sebuah array dari np dengan menggunakan fungsi list untuk mengkoleksikan data yang dipilih dan dapat diubah. Didalamnya diterapkan fungsi map untuk mengembalikan iterator dari datanya dengan memfungsikan lamda pada row dengan parameter[1] untuk membuat objek fungsi menjadi lebih kecil dan mudah dieksekusi dari variabel train.
\item Baris Code 5	: Membuat variabel test\_output dengan fungsi yang sama seperti train\_output yang membedakan hanya datanya / inputan yang diproses berasal dari variabel test
\end{enumerate}
\par
\item Hasil : \ref{chapter-7-in-4-fadila}
\par
\par
\begin{figure}[!hbtp]
\centering
\includegraphics[scale=0.4]{figures/chapter-7-in-4-fadila.jpg}
\caption{Code Program Pada In [4] - fadila}
\label{chapter-7-in-4-fadila}
\end{figure}
\par
\par
\end{itemize}
\par
\par
\par
\item Jelaskan Kode Program Pada Blok \# In[5]. Jelaskan Arti Dari Setiap Baris Kode Yang Dibuat Dan Hasil Keluarannya Dari Komputer Sendiri.
\begin{itemize}
\item Code Yang Digunakan : \ref{lst:chapter-7-5-fadila}.
\lstinputlisting[firstline=1, lastline=19,caption=File chapter-7-5-fadila.py, label={lst:chapter-7-5-fadila}]{src/1164072/praktek/chapter-7-5-fadila.py}
\par
\par
\item Penjelasan Code Perbaris	: 
\begin{enumerate}
\item Baris Code 1	: Memasukkan modul / fungsi LabelEncoder dari sklearn.processing yang digunakan untuk dapat  menormalkan label dimana label encoder hanya didefinisikan dengan nilai antara 0 dan -1.
\item Baris Code 2	: Memasukkan modul / fungsi OneHotEncoder dari sklearn.processing yang digunakan untuk mendefinisikan fitur input dimana mengambil nilai dalam kisaran [0, maks (nilai)).
\end{enumerate}
\par
\item Hasil : \ref{chapter-7-in-5-fadila}
\par
\par
\begin{figure}[!hbtp]
\centering
\includegraphics[scale=0.4]{figures/chapter-7-in-5-fadila.jpg}
\caption{Code Program Pada In [5] - fadila}
\label{chapter-7-in-5-fadila}
\end{figure}
\par
\par
\end{itemize}
\par
\par
\par
\item Jelaskan Kode Program Pada Blok \# In[6]. Jelaskan Arti Dari Setiap Baris Kode Yang Dibuat Dan Hasil Keluarannya Dari Komputer Sendiri.
\begin{itemize}
\item Code Yang Digunakan : \ref{lst:chapter-7-6-fadila}.
\lstinputlisting[firstline=1, lastline=19,caption=File chapter-7-6-fadila.py, label={lst:chapter-7-6-fadila}]{src/1164072/praktek/chapter-7-6-fadila.py}
\par
\par
\item Penjelasan Code Perbaris	: 
\begin{enumerate}
\item Baris Code 1	: Membuat variabel label\_encoder dengan penerapan modul / fungsi dari LabelEncoder tanpa parameter
\item Baris Code 2	: Membuat variabel integer\_encoded dengan penerapan fungsi label\_encoder.fit\_transform (ekstrasi fitur object ) dari variabel classes yang akan mengembalikan beberapa data yang diubah kembali.
\end{enumerate}
\par
\item Hasil : \ref{chapter-7-in-6-fadila}
\par
\par
\begin{figure}[!hbtp]
\centering
\includegraphics[scale=0.4]{figures/chapter-7-in-6-fadila.jpg}
\caption{Code Program Pada In [6] - fadila}
\label{chapter-7-in-6-fadila}
\end{figure}
\par
\par
\end{itemize}
\par
\par
\par
\item Jelaskan Kode Program Pada Blok \# In[7]. Jelaskan Arti Dari Setiap Baris Kode Yang Dibuat Dan Hasil Keluarannya Dari Komputer Sendiri.
\begin{itemize}
\item Code Yang Digunakan : \ref{lst:chapter-7-7-fadila}.
\lstinputlisting[firstline=1, lastline=19,caption=File chapter-7-7-fadila.py, label={lst:chapter-7-7-fadila}]{src/1164072/praktek/chapter-7-7-fadila.py}
\par
\par
\item Penjelasan Code Perbaris	: 
\begin{enumerate}
\item Baris Code 1	: Membuat variabel onehot\_encoder yang memanggil fungsi OneHotEncoder tanpa mengembalikan matriks karena sparse=false.
\item Baris Code 2	: Membuat variabel integer\_encoded memanggil variabel integer\_encoded pada kode program 6 untuk dieksekusi memberikan bentuk baru ke array tanpa mengubah datanya dari mengembalikan panjang nilai dari integer\_encoded.
\item Baris Code 3	: Onehotencoding melakukan fitting pada integer\_encoded.
\end{enumerate}
\item Hasil : \ref{chapter-7-in-7-fadila}
\par
\par
\begin{figure}[!hbtp]
\centering
\includegraphics[scale=0.4]{figures/chapter-7-in-7-fadila.jpg}
\caption{Code Program Pada In [7] - fadila}
\label{chapter-7-in-7-fadila}
\end{figure}
\par
\par
\end{itemize}
\par
\par
\par
\item Jelaskan Kode Program Pada Blok \# In[8]. Jelaskan Arti Dari Setiap Baris Kode Yang Dibuat Dan Hasil Keluarannya Dari Komputer Sendiri.
\begin{itemize}
\item Code Yang Digunakan : \ref{lst:chapter-7-8-fadila}.
\lstinputlisting[firstline=1, lastline=19,caption=File chapter-7-8-fadila.py, label={lst:chapter-7-8-fadila}]{src/1164072/praktek/chapter-7-8-fadila.py}
\par
\par
\item Penjelasan Code Perbaris	: 
\begin{enumerate}
\item Baris Code 1	: Mendefinisikan dan Membuat variabel train\_output\_int yang mengeksekusi label\_encoder dengan mengubah nilai dari parameter variabel train\_output.
\item Baris Code 2	: Mendefinisikan dan Membuat variabel train\_output yang mengeksekusi variabel onehot\_encoder dari kode program 7 dengan mengubah nilai dari variabel parameter train\_output\_int yang datanya sudah diubah kedalam bentuk array dan panjang nilai dari train\_output\_int telah dikembalikan.
\item Baris Code 3	: Mendefinisikan dan Membuat variabel test\_output\_int yang mengeksekusi label\_encoder dengan mengubah nilai dari parameter variabel test\_output.
\item Baris Code 4	: Mendefinisikan dan Membuat variabel test\_output yang mengeksekusi variabel onehot\_encoder dari kode program 7 dengan mengubah nilai dari variabel parameter test\_output\_int yang datanya sudah diubah kedalam bentuk array dan panjang nilai dari test\_output\_int telah dikembalikan
\item Baris Code 5	: Mendefinisikan dan Membuat variabel num\_classes untuk mengetahui jumlah class dari lebel\_encoder
\item Baris Code 6	: Mencetak dan menampilkan hasil dari variabel num\_classes menggunakan perintah print
\end{enumerate}
\item Hasil : \ref{chapter-7-in-8-fadila}
\par
\par
\begin{figure}[!hbtp]
\centering
\includegraphics[scale=0.4]{figures/chapter-7-in-8-fadila.jpg}
\caption{Code Program Pada In [8] - fadila}
\label{chapter-7-in-8-fadila}
\end{figure}
\par
\par
\end{itemize}
\par
\par
\par
\item Jelaskan Kode Program Pada Blok \# In[9]. Jelaskan Arti Dari Setiap Baris Kode Yang Dibuat Dan Hasil Keluarannya Dari Komputer Sendiri.
\begin{itemize}
\item Code Yang Digunakan : \ref{lst:chapter-7-9-fadila}.
\lstinputlisting[firstline=1, lastline=19,caption=File chapter-7-9-fadila.py, label={lst:chapter-7-9-fadila}]{src/1164072/praktek/chapter-7-9-fadila.py}
\par
\par
\item Penjelasan Code Perbaris	: 
\begin{enumerate}
\item Baris Code 1	: Difungsikan pemanggilan untuk melakukan importing fungsi model sequential dari library keras.
\item Baris Code 2	: Difungsikan pemanggilan untuk melakukan importing fungsi layer dense, dropout, dan flatten dari library keras.
\item Baris Code 3	: Difungsikan pemanggilan untuk  melakukan importing fungsi layer Conv2D dan MaxPooling2D dari library keras.
\end{enumerate}
\par
\item Hasil : \ref{chapter-7-in-9-fadila}
\par
\par
\begin{figure}[!hbtp]
\centering
\includegraphics[scale=0.4]{figures/chapter-7-in-9-fadila.jpg}
\caption{Code Program Pada In [9] - fadila}
\label{chapter-7-in-9-fadila}
\end{figure}
\par
\par
\end{itemize}
\par
\par
\par
\item Jelaskan Kode Program Pada Blok \# In[10]. Jelaskan Arti Dari Setiap Baris Kode Yang Dibuat Dan Hasil Keluarannya Dari Komputer Sendiri.
\begin{itemize}
\item Code Yang Digunakan : \ref{lst:chapter-7-10-fadila}.
\lstinputlisting[firstline=1, lastline=19,caption=File chapter-7-10-fadila.py, label={lst:chapter-7-10-fadila}]{src/1164072/praktek/chapter-7-10-fadila.py}
\par
\par
\item Penjelasan Code Perbaris	: 
\begin{enumerate}
\item Baris Code 1	: Merealisasikan dan Mengfungsikan pemodelan Sequential
\item Baris Code 2	: Menambahkan fungsi konvolusi 2D dengan 32 filter konvolusi masing-masing berukuran 3x3 dengan algoritam activation relu sesuai dengan ketentuan
\item Baris Code 3	: Dilanjutkan dengan data dari train\_input mulai dari baris nol dari fungsi input\_shape
\item Baris Code 4	: Melakukan penambahan fungsi max pooling dengan matriks 2x2 terhadap data
\item Baris Code 5	: Dilakukan lagi penambahkan Konvolusi 2D dengan 32 filter konvolusi masing-masing berukuran 3x3 dengan algoritam activation relu.
\item Baris Code 6	: Memfungsikan / Menambahkan kembali fungsi max pooling dengan matriks 2x2
\item Baris Code 7	: Memfungsikan flatten untuk mengembalikan salinan array asli dari data
\item Baris Code 8	: Mendefinisikan inputan dengan 1024 neuron dan menggunakan algoritma tanh untuk aktivasinya ( activation )
\item Baris Code 9	: Mendefinisikan fungsi Dropout dimana terdiri dari pengaturan secara acak tingkat pecahan unit input ke 0 pada setiap pembaruan selama waktu pelatihan, yang membantu mencegah overfitting sebesar 50\%  yang dituliskan dengan nilai 0,5.
\item Baris Code 10	: Untuk output layer menggunakan data dari variabel num\_classes dengan fugsi activationnya softmax.
\item Baris Code 11	: Melakukan konfigurasi dari proses pembelajaran, yang dilakukan melalui metode compile,sebelum melatih suatu model.
\item Baris Code 12	: Mencetak dan menampilkan representasi ringkasan model yang telah dibuat berdasrkan fungsi-fungsi yang diterapkan
\end{enumerate}
\par
\item Hasil : \ref{chapter-7-in-10-fadila}
\par
\par
\begin{figure}[!hbtp]
\centering
\includegraphics[scale=0.4]{figures/chapter-7-in-10-fadila.jpg}
\caption{Code Program Pada In [10] - fadila}
\label{chapter-7-in-10-fadila}
\end{figure}
\par
\par
\end{itemize}
\par
\par
\par
\item Jelaskan Kode Program Pada Blok \# In[11]. Jelaskan Arti Dari Setiap Baris Kode Yang Dibuat Dan Hasil Keluarannya Dari Komputer Sendiri.
\begin{itemize}
\item Code Yang Digunakan : \ref{lst:chapter-7-11-fadila}.
\lstinputlisting[firstline=1, lastline=19,caption=File chapter-7-11-fadila.py, label={lst:chapter-7-11-fadila}]{src/1164072/praktek/chapter-7-11-fadila.py}
\par
\par
\item Penjelasan Code Perbaris	: 
\begin{enumerate}
\item Baris Code 1	: Memasukkan / Mengimport library keras.callbacks dimana digunakan dalam penulisan log untuk TensorBoard, yang memungkinkan untuk memvisualisasikan grafik dinamis dari pelatihan dan metrik pengujian.
\item Baris Code 2	: Membuat variabel tenserboard yang mendefinisikan fungsi TensorBoard pada keras.callbacks yang digunakan sebagai alat visualisasi yang disediakan dengan TensorFlow. Kemudian untuk fungsi log\_dir (jalur direktori tempat menyimpan file log yang akan diuraikan oleh TensorBoard) memanggil data yaitu './logs/mnist-style'
\end{enumerate}
\par
\item Hasil : \ref{chapter-7-in-11-fadila}
\par
\par
\begin{figure}[!hbtp]
\centering
\includegraphics[scale=0.4]{figures/chapter-7-in-11-fadila.jpg}
\caption{Code Program Pada In [11] - fadila}
\label{chapter-7-in-11-fadila}
\end{figure}
\par
\par
\end{itemize}
\par
\par
\par
\item Jelaskan Kode Program Pada Blok \# In[12]. Jelaskan Arti Dari Setiap Baris Kode Yang Dibuat Dan Hasil Keluarannya Dari Komputer Sendiri.
\begin{itemize}
\item Code Yang Digunakan : \ref{lst:chapter-7-12-fadila}.
\lstinputlisting[firstline=1, lastline=19,caption=File chapter-7-12-fadila.py, label={lst:chapter-7-12-fadila}]{src/1164072/praktek/chapter-7-12-fadila.py}
\par
\par
\item Penjelasan Code Perbaris	: 
\begin{enumerate}
\item Baris Code 1	: Menerapkan fungsi model.fit yang didalamnya memproses train\_input, train\_output
\item Baris Code 2	: Selanjutnya pada penerapan fungsi yang sama difungsikan batch\_size ( jumlah sampel per pembaharuan sampel dari data yang diolah) apabila batch\_sizenya tidak ditemukan maka otomatis akan dijadikan nilai 32
\item Baris Code 3	: Pada penerapan fungsi yang sama, difungsikan epochs dimana perulangan dari berapa kali nilai yang digunakan untuk data, dan jumlahnya ialah 10
\item Baris Code 4	: Mendefinisikan fungsi verbose dimana digunakan sebagai opsi untuk menghasilkan informasi logging dari data yang ditentukan dengan nilai 2
\item Baris Code 5	: Mendefinisikan fungsi validation\_split untuk memecah nilai dari perhitungan validasinya sebesar 0,2. (Fraksi data pelatihan untuk digunakan sebagai data validasi)
\item Baris Code 6	: Mendefinisikan fungsi callsbacks dengan parameternya yang mengeksekusi tensorboard dimana digunakan untuk Visualisasikan parameter training, metrik, hiperparameter pada nilai/data yang diproses.
\item Baris Code 7	: Mendefinisikan variabel score dengan fungsi evaluate dari model yang ada dengan parameter test\_input, tst\_output dan verbose=2 dimana memprediksi output untuk input yang diberikan dan kemudian menghitung fungsi metrik yang ditentukan dalam modelnya
\item Baris Code 8	: Mencetak score optimasi dari test dengan ketentuan nilai parameter 0
\item Baris Code 9	: Mencetak score akurasi dari test dengan ketentuan nilai parameter 1
\end{enumerate}
\par
\item Hasil : \ref{chapter-7-in-12-fadila}
\par
\par
\begin{figure}[!hbtp]
\centering
\includegraphics[scale=0.4]{figures/chapter-7-in-12-fadila.jpg}
\caption{Code Program Pada In [12] - fadila}
\label{chapter-7-in-12-fadila}
\end{figure}
\par
\par
\end{itemize}
\par
\par
\par
\item Jelaskan Kode Program Pada Blok \# In[13]. Jelaskan Arti Dari Setiap Baris Kode Yang Dibuat Dan Hasil Keluarannya Dari Komputer Sendiri.
\begin{itemize}
\item Code Yang Digunakan : \ref{lst:chapter-7-13-fadila}.
\lstinputlisting[firstline=1, lastline=19,caption=File chapter-7-13-fadila.py, label={lst:chapter-7-13-fadila}]{src/1164072/praktek/chapter-7-13-fadila.py}
\par
\par
\item Penjelasan Code Perbaris	: 
\begin{enumerate}
\item Baris Code 1	: Melakukan impor modul time dari python anaconda
\item Baris Code 2	: Membuat variabel result berisikan array kosong.
\item Baris Code 3	: Menggunakan convolution 2D yang dimana akan memiliki 1 atau 2 layer.
\item Baris Code 4	: Mendefinisikan dan memfungsikan dense\_size dengan ukuran 128, 256, 512, 1024, 2048
\item Baris Code 5	: Mendefinsikan dan memfungsikan drop\_out dengan 0, 25\%, 50\%, dan 75\%
\item Baris Code 6	: Menerapkan dan Melakukan pemodelan Sequential
\item Baris Code 7	: Perlu memasukkan bentuk input apabila data tersebut merupakan layer pertama
\item Baris Code 8	: Apabila tidak maka hanya menambahkan layer
\item Baris Code 9	: Selanjutnya, ketika penambahan layer konvolusi selesai, maka dilakukan hal yang sama dengan max pooling.
\item Baris Code 10	: Kemudian meratakan atau flatten pada data dan menambahkan dense size ukuran apa pun yang berasal dari dense\_size, dimana menggunakan algoritma tanh
\item Baris Code 11	: Apabila dropout digunakan, dilakukan penambahan layer dropout. Untuk droupout berikut katakanlah 50\%, bahwa setiap kali ia memperbarui bobot setelah setiap batch, ada peluang 50\% untuk setiap bobot yang tidak akan diperbarui pada data tersebut
\item Baris Code 12	: Melakukan penempatan hasil drouput  di antara dua lapisan padat untuk dihidupkan dari melindunginya dari overfitting.
\item Baris Code 13	: Lapisan terakhir akan selalu menjadi jumlah kelas karena itu harus, dan menggunakan softmax. Itu dikompilasi dengan cara yang sama.
\item Baris Code 14	: Mengatur direktori log yang berbeda untuk TensorBoard sehingga dapat membedakan konfigurasi yang berbeda.
\item Baris Code 15	: Membuat variabel start yang akan memanggil modul time atau waktu
\item Baris Code 16	: Melakukan dan memfungsikan fit atau compile 
\item Baris Code 17	: Melakukan dan memfungsikan scoring dengan .evaluate yang akan menampilkan data loss dan accuracy dari model
\item Baris Code 18	: Untuk end merupakan variabel untuk melihat waktu akhir pada saat pemodelan berhasil dilakukan.
\item Baris Code 19	: Mencetak dan Menampilkan hasil dari run skrip berdasarkan fungsi-fungsi yang telah diterapkan diatas
\end{enumerate}
\par
\par
\item Hasil : \ref{chapter-7-in-13-fadila}
\par
\par
\begin{figure}[!hbtp]
\centering
\includegraphics[scale=0.4]{figures/chapter-7-in-13-fadila.png}
\caption{Code Program Pada In [13] - fadila}
\label{chapter-7-in-13-fadila}
\end{figure}
\par
\par
\par
\end{itemize}
\par
\par
\par
\item Jelaskan Kode Program Pada Blok \# In[14]. Jelaskan Arti Dari Setiap Baris Kode Yang Dibuat Dan Hasil Keluarannya Dari Komputer Sendiri.
\begin{itemize}
\item Code Yang Digunakan : \ref{lst:chapter-7-14-fadila}.
\lstinputlisting[firstline=1, lastline=19,caption=File chapter-7-14-fadila.py, label={lst:chapter-7-14-fadila}]{src/1164072/praktek/chapter-7-14-fadila.py}
\par
\par
\item Penjelasan Code Perbaris	: 
\begin{enumerate}
\item Baris Code 1	: Memfungsikan dan melakukan pemodelan Sequential pada code
\item Baris Code 2	: Memfungsikan dan menambahkan Convolution 2D dengan dmensi 32 untuk layer pertama,dan ukuran matriks 3x3 dengan function aktivasi yang digunakan yaitu relu dan menampilkan input\_shape
\item Baris Code 3	: Memfungsikan max pooling 2D dengan ukuran matriks 2x2
\item Baris Code 4	: Melakukan Convolusi lagi dengan kriteria yang sama tanpa menambahkan input untuk di layer kedua, hal ini dilakukan untuk mendapatkan data yang terbaik dari yang telah diproses
\item Baris Code 5	: Menambahkan fungsi Max pooling 2D dengan ukuran poolnya yaitu 2x2.
\item Baris Code 6	: Mendefinisikan fungsi flatten dimana diugnakan untuk meratakan inputan yang ada dan mengembalikan salinan bilangan / array data
\item Baris Code 7	: Menambahkan dense input sebanyak 128 neuron dengan menggunakan function aktivasi tanh.
\item Baris Code 8	: Menambahkan fungsi dropout sebanyak 50\% untuk menghindari overfitting pada data yang diproses
\item Baris Code 9	: Memfungsikan dense pada model untuk output, yang dimana layer berikut akan menjadi jumlah dari class yang ada.
\item Baris Code 10	: Melakukan compile / Mengcompile model yang telah dibuat dan didefinisikan sebelumnya ( diatas )
\item Baris Code 11	: Menampilkan dan mencetak ringkasan dari pemodelan yang dilakukan dari berbagai fungsi yang diterapkan.
\end{enumerate}
\par
\par
\item Hasil : \ref{chapter-7-in-14-fadila}
\par
\par
\begin{figure}[!hbtp]
\centering
\includegraphics[scale=0.4]{figures/chapter-7-in-14-fadila.png}
\caption{Code Program Pada In [14] - fadila}
\label{chapter-7-in-14-fadila}
\end{figure}
\par
\par
\end{itemize}
\par
\par
\par
\item Jelaskan Kode Program Pada Blok \# In[15]. Jelaskan Arti Dari Setiap Baris Kode Yang Dibuat Dan Hasil Keluarannya Dari Komputer Sendiri.
\begin{itemize}
\item Code Yang Digunakan : \ref{lst:chapter-7-15-fadila}.
\lstinputlisting[firstline=1, lastline=19,caption=File chapter-7-15-fadila.py, label={lst:chapter-7-15-fadila}]{src/1164072/praktek/chapter-7-15-fadila.py}
\par
\par
\item Penjelasan Code Perbaris	: ( Cuman ada 1 perintah namun pada penulisannya ada 3 baris )
\begin{enumerate}
\item Baris Code 1	: Memfungsikan dan Melakukan fit ( pencocokan data ) dengan join data train\_input dan test\_input agar dapat dilakukan pelatihan untuk jaringan pada smua data yang dimiliki.
\item Baris Code 2	: Memfungsikan kembali np.concenate ( join data ) dengan data dari train\_output dan test\_output
\item Baris Code 3	: Dilanjutkan dengan penerapan batch\_size sebesar 32, perulangan data 10 kali (epochs) dan verbose yang digunakan sebagai opsi untuk menghasilkan informasi logging dari data yang ditentukan dengan nilai 2
\end{enumerate}
\par
\par
\item Hasil : \ref{chapter-7-in-15-fadila}
\par
\par
\begin{figure}[!hbtp]
\centering
\includegraphics[scale=0.4]{figures/chapter-7-in-15-fadila.png}
\caption{Code Program Pada In [15] - fadila}
\label{chapter-7-in-15-fadila}
\end{figure}
\par
\par
\end{itemize}
\par
\par
\par
\item Jelaskan Kode Program Pada Blok \# In[16]. Jelaskan Arti Dari Setiap Baris Kode Yang Dibuat Dan Hasil Keluarannya Dari Komputer Sendiri.
\begin{itemize}
\item Code Yang Digunakan : \ref{lst:chapter-7-16-fadila}.
\lstinputlisting[firstline=1, lastline=19,caption=File chapter-7-16-fadila.py, label={lst:chapter-7-16-fadila}]{src/1164072/praktek/chapter-7-16-fadila.py}
\par
\par
\item Penjelasan Code Perbaris	: 
\begin{enumerate}
\item Baris Code 1	: Melakukan penyimpanan atau save model yang telah di latih dengan nama mathsymbols.model 
\end{enumerate}
\par
\par
\item Hasil : \ref{chapter-7-in-16-fadila}
\par
\par
\begin{figure}[!hbtp]
\centering
\includegraphics[scale=0.4]{figures/chapter-7-in-16-fadila.jpg}
\caption{Code Program Pada In [16] - fadila}
\label{chapter-7-in-16-fadila}
\end{figure}
\par
\par
\end{itemize}
\par
\par
\par
\item Jelaskan Kode Program Pada Blok \# In[17]. Jelaskan Arti Dari Setiap Baris Kode Yang Dibuat Dan Hasil Keluarannya Dari Komputer Sendiri.
\begin{itemize}
\item Code Yang Digunakan : \ref{lst:chapter-7-17-fadila}.
\lstinputlisting[firstline=1, lastline=19,caption=File chapter-7-17-fadila.py, label={lst:chapter-7-17-fadila}]{src/1164072/praktek/chapter-7-17-fadila.py}
\par
\par
\item Penjelasan Code Perbaris	: 
\begin{enumerate}
\item Baris Code 1	: Melakukan penyimpanan terhadap label enkoder (untuk membalikkan one-hot encoder) dengan nama classes.npy
\end{enumerate}
\par
\par
\item Hasil : \ref{chapter-7-in-17-fadila}
\par
\par
\begin{figure}[!hbtp]
\centering
\includegraphics[scale=0.4]{figures/chapter-7-in-17-fadila.jpg}
\caption{Code Program Pada In [17] - fadila}
\label{chapter-7-in-17-fadila}
\end{figure}
\par
\par
\end{itemize}
\par
\par
\par
\item Jelaskan Kode Program Pada Blok \# In[18]. Jelaskan Arti Dari Setiap Baris Kode Yang Dibuat Dan Hasil Keluarannya Dari Komputer Sendiri.
\begin{itemize}
\item Code Yang Digunakan : \ref{lst:chapter-7-18-fadila}.
\lstinputlisting[firstline=1, lastline=19,caption=File chapter-7-18-fadila.py, label={lst:chapter-7-18-fadila}]{src/1164072/praktek/chapter-7-18-fadila.py}
\par
\par
\item Penjelasan Code Perbaris	: 
\begin{enumerate}
\item Baris Code 1	: Memasukkan / mengimpor models dari librari Keras ( keras.models )
\item Baris Code 2	: Membuat variabel model2 yang akan difungsikan untuk memanggil model yang telah disave sebelumnya yaitu "mathsymbols.model"
\item Baris Code 3	: Menampilkan dan mencetak ringkasan dari hasil pemodelan pada variabel model2
\end{enumerate}
\par
\par
\item Hasil : \ref{chapter-7-in-18-fadila}
\par
\par
\begin{figure}[!hbtp]
\centering
\includegraphics[scale=0.4]{figures/chapter-7-in-18-fadila.jpg}
\caption{Code Program Pada In [18] - fadila}
\label{chapter-7-in-18-fadila}
\end{figure}
\par
\par
\end{itemize}
\par
\par
\par
\item Jelaskan Kode Program Pada Blok \# In[19]. Jelaskan Arti Dari Setiap Baris Kode Yang Dibuat Dan Hasil Keluarannya Dari Komputer Sendiri.
\begin{itemize}
\item Code Yang Digunakan : \ref{lst:chapter-7-19-fadila}.
\lstinputlisting[firstline=1, lastline=19,caption=File chapter-7-19-fadila.py, label={lst:chapter-7-19-fadila}]{src/1164072/praktek/chapter-7-19-fadila.py}
\par
\par
\item Penjelasan Code Perbaris	: 
\begin{enumerate}
\item Baris Code 1	: Melakukan pemanggilan fungsi LabelEncoder
\item Baris Code 2	: Membuat variabel label\_encoder akan memanggil class yang disave sebelumnya yaitu "classes.npy"
\item Baris Code 3	: Menerapkan function predict yang akan mengubah gambar kedalam bentuk array
\item Baris Code 4 	: Pada fungsi prediksi dibuat variabel newing yang akan memproses perubahan format gambar menjadi array (bilangan)
\item Baris Code 5	: Untuk Variabel newing didefinisikan tidak sama dengan nilai 255.0
\item Baris Code 5	: Kemudian untuk variabel prediction akan melakukan prediksi untuk model2 dengan reshape variabel newimg dengan bentuk array 4D yaitu ( 1, 32, 32 dan 3 ).
\item Baris Code 5	: Membuat variabel inverted yang akan mencari nilai tertinggi output dari hasil prediksi yang dilakukan sebelumnya
\item Baris Code 6	: Menampilkan dan mencetak hasil dari variabel prediction dan inverted
\end{enumerate}
\par
\par
\item Hasil : \ref{chapter-7-in-19-fadila}
\par
\par
\begin{figure}[!hbtp]
\centering
\includegraphics[scale=0.4]{figures/chapter-7-in-19-fadila.jpg}
\caption{Code Program Pada In [19] - fadila}
\label{chapter-7-in-19-fadila}
\end{figure}
\par
\par
\end{itemize}
\par
\par
\par
\item Jelaskan Kode Program Pada Blok \# In[20]. Jelaskan Arti Dari Setiap Baris Kode Yang Dibuat Dan Hasil Keluarannya Dari Komputer Sendiri.
\begin{itemize}
\item Code Yang Digunakan : \ref{lst:chapter-7-20-fadila}.
\lstinputlisting[firstline=1, lastline=19,caption=File chapter-7-20-fadila.py, label={lst:chapter-7-20-fadila}]{src/1164072/praktek/chapter-7-20-fadila.py}
\par
\par
\item Penjelasan Code Perbaris	: 
\begin{enumerate}
\item Baris Code 1	: Menerapkan dan Melakukan prediksi dari pelatihan dari gambar v2-00010.png
\item Baris Code 2	: Menerapkan dan Melakukan prediksi dari pelatihan dari gambar v2-00500.png
\item Baris Code 3	: Menerapkan dan Melakukan prediksi dari pelatihan dari gambar v2-00700.png
\end{enumerate}
\par
\item Hasil : \ref{chapter-7-in-20-fadila}
\par
\par
\begin{figure}[!hbtp]
\centering
\includegraphics[scale=0.4]{figures/chapter-7-in-20-fadila.jpg}
\caption{Code Program Pada In [20] - fadila}
\label{chapter-7-in-20-fadila}
\end{figure}
\par
\par
\end{itemize}
\par
\par
\par
\end{enumerate}


\subsection{Penanganan Error}
Penjelasan Tugas Harian 12 ( No 1-20 ).
\begin{enumerate}
\item Error 1	:
\begin{itemize}
\item Screenshoot Error 	: \ref{chapter-7-error-1-fadila}
\par
\par
\begin{figure}[!hbtp]
\centering
\includegraphics[scale=0.2]{figures/chapter-7-error-1-fadila.jpg}
\caption{Error 1 - fadila}
\label{chapter-7-error-1-fadila}
\end{figure}
\par
\item Code Error		:
\begin{lstlisting}
ModuleNotFoundError: No module named 'tensorflow'
\end{lstlisting}
\item Penanganan Error	:
\begin{enumerate}
\item Pertama-tama pastikan salahnya seperti apa ( model dan jenis errornya )
\item Kemudian, silahkan proses error tersebut dengan cara yang sesuai
\item Berdasarkan error maka penyelesaiannya ialah melakukan instalasi terhadap module 'tensorflow' sehingga code dapat dijalankan
\item Buka Anaconda Prompt kemudian lakukan perintah seperti pada gambar berikut ( conda install -c conda-forge tensorflow ) : \ref{chapter-7-penanganan-error-1-fadila}
\par
\begin{figure}[!hbtp]
\centering
\includegraphics[scale=0.2]{figures/chapter-7-penanganan-error-1-fadila.jpg}
\caption{Penanganan Error 1 - fadila}
\label{chapter-7-penanganan-error-1-fadila}
\end{figure}
\par
\item Setelah melakukan perintah berikut maka silahkan jalankan kembali code maka tidak akan terjadi lagi error tersebut.
\end{enumerate}
\end{itemize}
\par
\par
\item Error 2 :
\begin{itemize}
\item Screenshoot Error 	: \ref{chapter-7-error-2-fadila}
\par
\par
\begin{figure}[!hbtp]
\centering
\includegraphics[scale=0.2]{figures/chapter-7-error-2-fadila.jpg}
\caption{Error 2 - fadila}
\label{chapter-7-error-2-fadila}
\end{figure}
\par
\item Code Error		:
\begin{lstlisting}
NameError: name 'model' is not defined
\end{lstlisting}
\item Penanganan Error	:
\begin{enumerate}
\item Pertama-tama pastikan salahnya seperti apa ( model dan jenis errornya )
\item Kemudian, silahkan proses error tersebut dengan cara yang sesuai
\item Berdasarkan error maka penyelesaiannya ialah melakukan pendefinisian variabel model sehingga code dapat dijalankan 
\par
\begin{figure}[!hbtp]
\centering
\includegraphics[scale=0.2]{figures/chapter-7-penanganan-error-2-fadila.jpg}
\caption{Penanganan Error 2 - fadila}
\label{chapter-7-penanganan-error-2-fadila}
\end{figure}
\par
\item Setelah melakukan pendefinisian variabel tersebut maka silahkan jalankan kembali code maka tidak akan terjadi lagi error tersebut.
\end{enumerate}
\end{itemize}
\end{enumerate}
\par
\par